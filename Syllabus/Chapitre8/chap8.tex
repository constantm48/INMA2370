\ifx \globalmark \undefined %% This is default.
	\documentclass[twoside,openright,11pt,a4paper]{report}

%\compiler avec xelatex
%\usepackage[applemac]{inputenc}
\usepackage[T1]{fontenc}
\usepackage[utf8]{inputenc} %latin1 est possible
%\usepackage[latin1]{inputenc} %latin1 est possible
\usepackage[francais]{babel}
\usepackage{lettrine}

\usepackage[text={13cm,20cm},centering]{geometry}

\renewcommand{\familydefault}{cmss}

\usepackage{graphicx}
\usepackage{amsmath}
\usepackage{amsfonts}
\usepackage{amssymb}
\usepackage{amsthm}
\usepackage{bm}
\usepackage{color}

\newcommand{\real}{\mathbb{R}}
\newcommand{\mb}{\mathbf}
\newcommand{\bos}{\boldsymbol}

\def \RR {I \! \! R}

\newcommand{\e}{\begin{equation}}  
\newcommand{\ee}{\end{equation}}
\newcommand{\eqn}{\begin{eqnarray}} 
\newcommand{\eeqn}{\end{eqnarray}} 
\newcommand{\eqnn}{\begin{eqnarray*}} 
\newcommand{\eeqnn}{\end{eqnarray*}} 

\newcommand{\bpm}{\begin{pmatrix}}
\newcommand{\epm}{\end{pmatrix}}

%\newcommand{\{\c c}}{\c c}

\newcommand{\bma}{\left(\begin{array}}
\newcommand{\ema}{\end{array}\right)} 
\newcommand{\hh}{\hspace{2mm}}
\newcommand{\hd}{\hspace{5mm}}
\newcommand{\hu}{\hspace{1cm}}
\newcommand{\vv}{\vspace{2mm}}
\newcommand{\vd}{\vspace{5mm}}
\newcommand{\vm}{\vspace{-2mm}}
\newcommand{\teq}{\triangleq}
%\newcommand{\qedb}{\,$\Box$}
\newcommand{\blanc}{$\left. \right.$}
\newcommand{\frts}[2]%
         {\frac{{\textstyle #1}}{{\textstyle #2}}}

\newcommand{\bindex}[3]%
{
\renewcommand{\arraystretch}{0.5}
\begin{array}[t]{c}
#1\\
{\scriptstyle #2}\\
{\scriptstyle #3}
\end{array}
\renewcommand{\arraystretch}{1}
}

\theoremstyle{definition}
\newtheorem{exemple}{{\bf Example}}[chapter]
\newtheorem{theoreme}[exemple]{{\bf Theorem}}
\newtheorem{propriete}[exemple]{{\bf Property}}
\newtheorem{definition}[exemple]{{\bf Definition}}
\newtheorem{remarque}[exemple]{{\bf Remark}}
\newtheorem{remarques}[exemple]{{\bf Remarks}}
\newtheorem{lemme}[exemple]{{\bf Lemma}}
\newtheorem{hypothese}[exemple]{{\bf Hypothesis}}
\newtheorem{exercice}{{\bf Exercise}}[chapter]

\newcommand{\xqedhere}[2]{%
 \rlap{\hbox to#1{\hfil\llap{\ensuremath{#2}}}}}

\newcommand{\xqed}[1]{%
 \leavevmode\unskip\penalty9999 \hbox{}\nobreak\hfill
 \quad\hbox{\ensuremath{#1}}}

\newcommand{\gf}{\fg\,\,}

\newcommand{\cata}[1] %
     {\renewcommand{\arraystretch}{0.5}
     \begin{array}[t]{c} \longrightarrow \\ {#1} \end{array}
     \renewcommand{\arraystretch}{1}}

\usepackage[isu]{caption}
%\usepackage[font=small,format=plain,labelfont=bf,up,textfont=it,up]{caption}
\setlength{\captionmargin}{60pt}

\newcommand{\cqfd}
{%
\mbox{}%
\nolinebreak%
\hfill%
\rule{2mm}{2mm}%
\medbreak%
\par%
}

\pagestyle{headings}

\renewcommand{\sectionmark}[1]{%
\markright{\thesection.\ #1}{}}

\renewcommand{\chaptermark}[1]{%
\markboth{\chaptername\ \thechapter.\ #1}{}}

\makeatletter 
\def\@seccntformat#1{\csname the#1\endcsname.\;} 
\makeatother

\title{ {\Huge {\textbf{Modélisation et analyse  \\ \vspace{4mm} des systèmes dynamiques }}} \\ \vspace{4cm} G. Bastin}

\date{\today}
	\begin{document} %% Crashes if put after (one of the many mysteries of LaTeX?).
\else 
	\documentclass{standalone}
	\begin{document}
\fi

\graphicspath{ {Chapitre8/images/} }

\setcounter{chapter}{7}
\chapter{Systèmes plans}
\chaptermark{Systèmes plans}\label{sysplans}


\lettrine[lines=1]{\bf D}{}ans ce chapitre, nous étudions en détail le comportement des  trajectoires des syst{è}mes
dynamiques de dimension 2 (appel{é}s aussi syst{è}mes plans) lorsque l'entr{é}e $u(t)$ est constante~: $u(t)=\bar u$.  Ces syst{è}mes sont d{é}crits par les {é}quations
suivantes~:
\eqnn
\dot x_1&=&f_1(x_1,x_2,\bar u),\\
\dot x_2&=&f_2(x_1,x_2,\bar u).
\eeqnn
Une importante motivation de cette restriction aux syst{è}mes plans est d'illustrer facilement les r{é}sultats obtenus en repr{é}sentant les
orbites dans le {\em plan de phase}, c.{à}.d. le plan des
variables d'{é}tat $x_1$ et $x_2$. En outre, les syst{è}mes plans permettent d'illustrer la plupart des comportements caract{é}ristiques qui diff{é}rencient les syst{è}mes non lin{é}aires des syst{è}mes lin{é}aires.

Nous
{é}tudierons successivement les trajectoires des syst{è}mes
lin{é}aires, puis le comportement des trajectoires des syst{è}mes non
lin{é}aires au voisinage des points d'{é}quilibre. Ensuite, nous nous
int{é}resserons aux trajectoires p{é}riodiques et aux cycles limites,
pour conclure par un aper\c cu de la th{é}orie des bifurcations.

\section{Systèmes linéaires plans}

Considérons les
syst{è}mes lin{é}aires plans lorsque l'entr{é}e $u(t)$ est
constante~: $u(t)=\bar u$.
Ces syst{è}mes sont repr{é}sent{é}s par l'{é}quation $$\dot x=Ax +B \bar u,$$ o{ù} $A$ est une matrice
$2\times 2$. Nous supposons qu'il existe au moins un {é}tat d'{é}quilibre $\bar x$ correspondant {à} $\bar u$.

Par une transformation d'{é}tat appropri{é}e, $z=M^{-1}(x-\bar x)$, on se ram{è}ne au
syst{è}me $$\dot z=A'z$$ o{ù} 
$$A'=M^{-1}AM.$$ 
Les valeurs propres de la matrice $A'$ sont celles de la matrice $A$ et elle possède l'une
des trois formes suivantes~:
\begin{itemize}
\item[{\bf a.}]
$$A'=\bma{cc} \lambda_1&0\\0&\lambda_2\ema$$ Cette forme correspond au cas
où la matrice $A$ a deux valeurs propres r{é}elles distinctes ou
une valeur propre r{é}elle double de multiplicit{é} g{é}om{é}trique 2. \\
\item[{\bf b.}]
$$A'=\bma{cc} \lambda&1\\0&\lambda\ema$$ Cette forme correspond au cas où la
matrice $A$ a une valeur propre r{é}elle double de multiplicit{é}
g{é}om{é}trique {é}gale {à} un. C'est la ``forme de Jordan'' associ{é}e {à}
$A$. \\
\item[{\bf c.}]
$$A'=\bma{cc} \alpha&\beta\\-\beta&\alpha\ema,\;\;\; \beta >0$$ Cette forme
correspond au cas où la matrice $A$ a deux valeurs propres
complexes conjugu{é}es $\alpha\pm\beta i$. \\
\end{itemize} 

Dans ces nouvelles coordonn{é}es, les trajectoires se calculent facilement et sont d{é}crites par les {é}quations suivantes :
\begin{itemize}
\item[{\bf a.}]
\eqnn
z_1(t)&=&z_1(0)e^{\lambda_1 t},\\
z_2(t)&=&z_2(0)e^{\lambda_2 t}.
\eeqnn 
\item[{\bf b.}]
\eqnn
z_1(t)&=&z_1(0)e^{\lambda t}+ t z_2(0)e^{\lambda t},\\
z_2(t)&=&z_2(0)e^{\lambda t}.
\eeqnn
\item[{\bf c.}]
\eqnn
z_1(t)&=&e^{\alpha t}(z_1(0)\cos \beta t+z_2(0) \sin \beta t),\\
z_2(t)&=&e^{\alpha t}(z_2(0)\cos \beta t-z_1(0) \sin \beta t).
\eeqnn
\end{itemize} 

Les tableaux 8.1 à 8.3 illustrent les orbites en fonction de l'une de ces
trois formes et en fonction du signe des valeurs propres. Ces orbites
sont repr{é}sent{é}es dans le plan $(z_1,z_2)$ et dans le plan 
$(x_1,x_2)$, centré au point d'équilibre $(\bar x_1, \bar x_2)$.
Dans ce deuxième cas, les directions privil{é}gi{é}es dans les
figures correspondent aux vecteurs propres de la matrice $A$.

\begin{table}
\hspace*{-5mm}
\begin{eqnarray*}
\begin{array}{|c|c|c|l|}
\hline
\mbox{Type}&\mbox{Allure des}& \mbox{Allure des} &\mbox{Conditions}\\&&& \mbox{sur les}\\
\mbox{de l'{é}quilibre}&\mbox{trajectoires }(z_1,z_2)&\mbox{trajectoires }(x_1,x_2)&
\mbox{valeurs propres}\\
\hline
\mbox{Noeud attractif} & & &\lambda_2 \leq \lambda_1 < 0\\
&\mbox{\includegraphics[width=25mm]{chap9taba1z}}&\mbox{\includegraphics[width=25mm]{chap9taba1x}} &\\
\hline
\mbox{Noeud répulsif}& & & 0<\lambda_1\leq\lambda_2 \\
&\mbox{\includegraphics[width=25mm]{chap9taba2z}}& \mbox{\includegraphics[width=25mm]{chap9taba2x}} &\\
\hline
\mbox{Col} & & & \lambda_1 < 0 < \lambda_2 \\
&\mbox{\includegraphics[width=25mm]{chap9taba3z}}& \mbox{\includegraphics[width=25mm]{chap9taba3x}}&\\
\hline
\mbox{Equilibre} && &\lambda _1 = 0,  \\
\mbox{non isol{é}, attractif}&&&\lambda_2 < 0\\
&\mbox{\includegraphics[width=25mm]{chap9taba4z}}&\mbox{\includegraphics[width=25mm]{chap9taba4x}} &\\
\hline
\mbox{Equilibre} && &\lambda _1 = 0, \\
\mbox{non isol{é}, répulsif}&&& \lambda_2 > 0 \\
&\mbox{\includegraphics[width=25mm]{chap9taba5z}}& \mbox{\includegraphics[width=25mm]{chap9taba5x}}&\\
\hline
\end{array}
\end{eqnarray*}
\caption{Orbites des systèmes linéaires plans~: cas\;{\em \bf a.}}
\label{tablea}
\end{table}

\begin{remarques}{\hspace{1cm}}\end{remarques}
\begin{enumerate}
\item Dans les deux premiers cas repris dans le tableau~\ref{tablea}, lorsque
$\lambda_1=\lambda_2$, les trajectoires sont rectilignes et peuvent donc
{ê}tre repr{é}sent{é}es par un faisceau de droites issu de l'origine.
\item Dans le cas ou l'une des deux valeurs propres est nulle, l'{é}quilibre
n'est pas isol{é}. Le vecteur propre correspondant {à} la valeur propre
nulle d{é}finit une droite de points d'{é}quilibre et toutes les
trajectoires sont rectilignes et convergent vers ou sont issues d'un point
de cette droite d'{é}quilibres.
\end{enumerate}
\begin{table}
\begin{eqnarray*}
\begin{array}{|c|c|c|l|}
\hline
\mbox{Type}&\mbox{Allure des}& \mbox{Allure des} &\mbox{Conditions sur}\\
\mbox{de l'{é}quilibre}&\mbox{trajectoires }(z_1,z_2)&\mbox{trajectoires }(x_1,x_2)&\mbox{les valeurs propres}\\
\hline
\begin{array}{l}
\mbox{Noeud d{é}g{é}n{é}r{é}}\\
\mbox{attractif}\end{array} & & &\begin{array}{l} \lambda = 0\\  \end{array} \mbox{(Jordan)} \\
&\mbox{\includegraphics[width=30mm]{chap9tabB1z}}&\mbox{\includegraphics[width=30mm]{chap9tabB1x}} &\\
\hline
\begin{array}{l}
\mbox{Noeud d{é}g{é}n{é}r{é}}\\   
\mbox{répulsif}\end{array} & & & \begin{array}{l}\lambda >0 \\
\end{array}\mbox{(Jordan)}\\
&\mbox{\includegraphics[width=30mm]{chap9tabB2z}}&\mbox{\includegraphics[width=30mm]{chap9tabB2x}} &\\
\hline
\begin{array}{l}
\mbox{Equilibre }\\   
\mbox{non-isolé}\end{array} & & & \begin{array}{l}\lambda=0\\
\end{array}\mbox{(Jordan)}\\
&\mbox{\includegraphics[width=30mm]{chap9tabB3z.jpg}}&\mbox{\includegraphics[width=30mm]{chap9tabB3x}} &\\
\hline
\end{array}
\end{eqnarray*}
\caption{Orbites des systèmes linéaires plans~: cas\;{\em \bf b.}}
\label{tableb}
\end{table}
%inserer ici les remarques sur le cas b)
\begin{table}
\begin{eqnarray*}
\begin{array}{|c|c|c|l|}
\hline
\mbox{Type}&\mbox{Allure des}& \mbox{Allure des} &\mbox{Conditions sur}\\
\mbox{de l'{é}quilibre}&\mbox{trajectoires }(z_1,z_2)&\mbox{trajectoires }(x_1,x_2)&\mbox{les valeurs propres}\\
\hline
\mbox{Foyer attractif} & & &\begin{array}{l}
\lambda_{1,2} = \alpha \pm \beta i\\
\alpha < 0 , \;\; \beta \neq 0 \end{array}\\
&\mbox{\includegraphics[width=30mm]
{chap9tabC1z}}&\mbox{\includegraphics[width=30mm]{chap9tabC1x}} &\\
\hline
\mbox{Foyer répulsif} & & &\begin{array}{l}
\lambda_{1,2} = \alpha \pm \beta i\\
\alpha >0 , \;\; \beta \neq 0 \end{array}\\
&\mbox{\includegraphics[width=30mm]{chap9tabC2z}}&\mbox{\includegraphics[width=30mm]{chap9tabC2x}} &\\
\hline
\mbox{Centre} & & &\begin{array}{l}
\lambda_{1,2} = \pm \beta i\\
 \beta \neq 0 \end{array}\\
&\mbox{\includegraphics[width=30mm]{chap9tabC3z}}& \mbox{\includegraphics[width=30mm]{chap9tabC3x}}&\\
\hline
\end{array}
\end{eqnarray*}
\caption{Orbites des systèmes linéaires plans~: cas\;{\em \bf c.}}
\label{tablec}
\end{table}
\renewcommand{\arraystretch}{1.0}

\begin{definition}{\em
 Lorsque l'{é}quilibre est tel que les trajectoires con\-ver\-gent vers cet
{é}quilibre, on dira qu'il s'agit d'un {é}quilibre {\em attractif}.}\cqfd
\end{definition}

Les valeurs propres $\lambda_1$ et $\lambda_2$ de la matrice $A$ sont les racines du polyn{\^o}me  caractéristique
\eqnn
p(x)&=&x^2-(\lambda_1+\lambda_2)x +\lambda_1 \lambda_2\\
&=&x^2-\mbox{tr}(A)x+\mbox{det}A.
\eeqnn
 Observons que pour
d{é}terminer l'allure des trajectoires, il n'est pas n{é}cessaire de
calculer explicitement ces valeurs propres. La figure~\ref{fig:figlambda12}
caract{é}rise la nature de l'{é}quilibre (et d{è}s lors l'allure des
trajectoires) en fonction des deux coefficients du polyn{\^o}me
caract{é}ristique respectivement {é}gaux {à} l'oppos{é} de la somme et au
produit des valeurs propres.

\begin{figure}[htbp]
   \centering
   \includegraphics[width=12cm]{figlambda12} 
   \caption{
   t{é}risation des {é}quilibres en fonction de la somme et du
produit des valeurs propres}
   \label{fig:figlambda12}
\end{figure}

On peut se demander dans quelle mesure la nature des trajectoires d{é}crites
ci-dessus est sensible {à} des perturbations du syst{è}me. Pour répondre à cette question, considérons un syst{è}me lin{é}aire nominal $\dot x=
A x+Bu$ et une perturbation du syst{è}me nominal de la forme $\dot x= (A+\Delta A) x+Bu$. Si la matrice $A$ poss{è}de des valeurs
propres {\em distinctes}, on peut montrer que celles-ci d{é}pendent contin{\^u}ment
des coefficients de $A$, ce qui signifie que pour tout nombre positif
$\epsilon$, il existe un nombre positif $\delta$ tel que si chacun des
coefficients de la perturbation $\Delta A$ est plus petit que $\delta$, les
valeurs propres de la matrice perturb{é}e $A+\Delta A$ seront {à}
l'int{é}rieur de boules de rayon $\epsilon$ centr{é}es en les valeurs
propres de $A$. Donc, toute valeur propre initialement {à} l'int{é}rieur du
demi-plan de gauche ($Re(\lambda)<0$) ou du demi-plan de droite
($Re(\lambda)>0$) restera dans le m{ê}me demi-plan pour des perturbations
$\Delta A$ suffisamment petites et, qualitativement, les trajectoires du
syst{è}me perturb{é} seront semblables {à} celles du syst{è}me nominal~:
un foyer
attractif reste un foyer attractif, un noeud répulsif reste un noeud répulsif, un col reste un col,... On dit dans ce cas que de tels syst{è}mes (ou de tels {é}quilibres) sont {\em
structurellement stables}. 

Il n'en va pas de m{ê}me dans le cas d'un
{é}quilibre de type {\em centre}, auquel correspondent des trajectoires
p{é}riodiques elliptiques et des valeurs propres imaginaires pures. Dans ce
cas en effet, la moindre perturbation de la matrice $A$ peut faire en sorte
que les valeurs propres quittent l'axe imaginaire et que les trajectoires
correspondantes deviennent un foyer attractif ou répulsif. Un syst{è}me
lin{é}aire auquel correspond un {é}quilibre de type {\em centre} n'est donc
{\bf pas} structurellement stable. 

Le cas de syst{è}mes lin{é}aires ayant
une ou deux valeurs propres nulles conduit {é}galement {à} un changement
qualitatif des trajectoires sous l'effet de perturbations arbitrairement
petites. Lorsque le syst{è}me poss{è}de une valeur propre double
diff{é}rente de $0$, de petites perturbations peuvent conduire {à} des valeurs
propres r{é}elles ou complexes conjugu{é}es, mais la localisation dans l'un
ou l'autre des demi-plans ne sera pas modifi{é}e. Un noeud attractif (répulsif)
d{é}g{é}n{é}r{é} peut donc se transformer en noeud attractif (répulsif) ou en
foyer attractif (répulsif).

L'analyse pr{é}c{é}dente montre bien que c'est l'axe imaginaire qui peut
poser probl{è}me. On introduit d{è}s lors la d{é}finition suivante. 

\begin{definition} \label{hyperbolique} 
Si toutes
les valeurs propres de $A$ ont une partie r{é}elle non nulle,  le syst{è}me
$\dot x=A x$ (ou le point d'{é}quilibre)
 est dit hyperbolique. \qed 
\end{definition}

Il r{é}sulte de ce qui pr{é}c{è}de qu'un syst{è}me hyperbolique est
structurellement stable et que les trajectoires resteront qualitativement
semblables pour de petites perturbations. Dans le cas
d'une valeur propre double diff{é}rente de z{é}ro, de petites perturbations peuvent engendrer soit un foyer, soit un noeud; 
mais le caract{è}re attractif ou r{é}pulsif de l'{é}quilibre sera lui de toute fa\c{c}on 
pr{é}serv{é}.
Ces consid{é}rations vont {ê}tre
de grande importance pour l'analyse des syst{è}mes non lin{é}aires.
 

 \section{Linéarisation des systèmes non linéaires}

Les orbites illustr{é}es dans les tableaux de la section pr{é}c{é}dente ne sont pas
seulement valables au voisinage du point d'{é}quilibre (ramen{é} {à}
l'origine). On a bien caract{é}ris{é} gr{\^a}ce {à} ces tableaux l'ensemble des
orbites possibles des syst{è}mes lin{é}aires plans, quelle que soit la
condition initiale. Cette observation constitue une diff{é}rence fondamentale
entre syst{è}mes lin{é}aires et non lin{é}aires. En effet, on a vu
 au chapitre pr{é}c{é}dent que les syst{è}mes non-lin{é}aires peuvent
pr{é}senter plusieurs {é}quilibres isol{é}s distincts pour une m{ê}me
valeur de l'entr{é}e $\bar u$. Ceci implique que, contrairement au cas
des syst{è}mes lin{é}aires, le comportement des orbites au voisinage
d'un {é}quilibre gardera le plus souvent un \textit{caract{è}re local} et ne pourra
nullement
{ê}tre {é}tendu {à} l'ensemble du plan de phase. Moyennant cette
restriction, un r{é}sultat important permet cependant d'{é}tendre  aux
syst{è}mes non lin{é}aires une partie de l'analyse que nous venons de
d{é}velopper pour les syst{è}mes lin{é}aires.

Soit le syst{è}me dynamique
d{é}crit par
\eqn
\dot x_1&=&f_1(x_1,x_2, \bar u),\label{f1}\\
\dot x_2&=&f_2(x_1,x_2, \bar u).\label{f2}
\eeqn
ou, sous forme condens{é}e,
\e \label{fxu}
\dot x=f(x,\bar u),
\ee pour lequel on suppose l'existence d'un {é}quilibre $(\bar x,\bar u)$ tel que $f(\bar x,\bar u)=0$.
 On suppose en outre que la fonction $f(x,\bar u)$ est suffisamment 
 régulière dans le voisinage de cet équilibre pour y admettre un développement de Taylor convergent: 
 $$ \dot x = f(\bar{x}, \bar{u}) + \left ( \frac{\partial f(x, \bar{u})}{ \partial x}\right)_ {\bar x} {(x - \bar{x})} + \cdots . $$
L'\textit{approximation lin{é}aire} de ce syst{è}me au voisinage de l'{é}quilibre 
$(\bar x,\bar u)$, obtenue en n{é}gligeant les termes
d'ordre sup{é}rieur ou {é}gal {à} 2 dans le
d{é}veloppement de Taylor de $f(x,\bar u)$ autour de $(\bar x,\bar u)$,
est donn{é}e par
\e \label{approxli}
\dot {\tilde x} = \left ( {\frac{{\textstyle \partial f(x, \bar u)}}{{\textstyle \partial x}}} \right )_{\bar x} \tilde x 
\ee
o{ù} $\tilde x=x-\bar x$. Notons $A=\left (\frac{\partial f(x, \bar u)}{\partial x}\right )_{\bar x}$, la matrice Jacobienne de $f$ {à} l'{é}quilibre.  On peut alors g{é}n{é}raliser la d{é}finition~\ref{hyperbolique} comme suit~:
\begin{definition}\label{hyperbnl} {\bf{\em Equilibre hyperbolique}}

L'{é}quilibre $(\bar x,\bar u)$ du syst{è}me non lin{é}aire (\ref{fxu}) est
dit hyperbolique si toutes les valeurs propres de $A$ ont une partie
r{é}elle non nulle ($Re(\lambda_i(A))$ $ \neq 0, \forall i)$.\qed 
\end{definition}

Il doit {ê}tre clair que c'est bien {\em l'{é}quilibre} $(\bar x,\bar u)$ qui
est (ou qui n'est pas) hyperbolique, et non le syst{è}me non lin{é}aire
(\ref{fxu}). En effet, ce syst{è}me peut avoir plusieurs {é}quilibres
isol{é}s pour une m{ê}me valeur $\bar u$, certains {é}tant hyperboliques et
d'autres non.  Dans quelle mesure l'{é}tude de l'approximation lin{é}aire d'un syst{è}me non-lin{é}aire au voisinage d'un {é}quilibre permet-elle d'en d{é}duire le comportement du syst{è}me non-lin{é}aire~? Pour pr{é}ciser ce que l'on entend par comportement, nous voulons pouvoir comparer les trajectoires et introduisons d{è}s lors la d{é}finition suivante.

\begin{definition} 
Les trajectoires (ou les orbites) de deux systèmes dynamiques sont 
{\em topologiquement équivalentes} s'il existe un {\em homéomorphisme}
 (une bijection bicontinue) qui permet de passer d'une trajectoire du premier système 
 à une trajectoire du second. \qed
\end{definition} 

\begin{theoreme} {\bf{\em Hartman-Grobman, 1959}}

Si l'{é}quilibre $(\bar x,\bar u)$ est hyperbolique, alors les 
trajectoires du syst{è}me non
lin{é}aire~(\ref{fxu})
{\em dans un voisinage de l'{é}quilibre $(\bar x,\bar u)$} sont
topologiquement {é}quivalentes {à} celles  de l'approximation
lin{é}aire~(\ref{approxli}). Plus précisément, il existe un voisinage $X$ de $\bar x$, un voisinage $\tilde{X}$ de $0$ et un homéomorphisme $h: X \to \tilde{X}$ avec $h(\bar x)=0$
tel que si $t \mapsto x(t)$ est une trajectoire du système non 
lin{é}aire~(\ref{fxu}) contenue dans $X$ (pour un certain intervalle de temps), alors $t \mapsto h(x(t))$ est une trajectoire du système linéaire (\ref{approxli}). \label{Hart}\qed
\end{theoreme}

Des trajectoires topologiquement équivalentes ont la m{ê}me allure. On pourra donc parler de noeud ou de foyer attractif ou répulsif, ou encore de
col, pour les {é}quilibres de syst{è}mes non lin{é}aires, en {é}tudiant les valeurs propres de la matrice de l'approximation lin{é}aire, mais {\bf pas} de
centre.

\begin{remarques}\hspace{10mm}\end{remarques}

\begin{enumerate}
\item L'int{é}r{ê}t de ce th{é}or{è}me est {é}vident. Sa limitation
principale, {à} savoir son caract{è}re local, ne l'est pas moins. En
particulier, ce th{é}or{è}me ne fournit aucune indication sur la taille du
bassin d'attraction d'un {é}quilibre attractif.
\item Dans le cas d'un {é}quilibre non hyperbolique, ce sont les termes
d'ordre sup{é}rieur, ceux-l{à} m{ê}me qui ont
{é}t{é} n{é}glig{é}s, qui d{é}termineront localement l'allure des
trajectoires.
\item Les outils développés jusqu'ici dans ce chapitre ne sont pas propres au systèmes plans. Classification des systèmes linéaires, linéarisation, théorème de Hartman-Grobman se généralisent sans problème en toute dimension
\item Dans la linéarisation~(\ref{approxli}), on garde $u=\bar u$ constant. On pourrait également linéariser $f$ autour de $u=\bar u$ pour obtenir un linéarisé de type $\tilde{x}=A\tilde{x} + B\tilde{u}$. Tant que $\tilde{u}=u-\bar u$ reste suffisamment petit, et pour un intervalle de temps suffisamment petit, les trajectoires des systèmes non linéaire et linéarisé resteront proches, mais il n'existe pas de variante simple du théorème de Hartman-Grobman dans ce cas. 
\cqfd
\end{enumerate}



\section{Au delà des systèmes plans}

Les considérations précédentes ne sont pas propres aux systèmes plans. 

Le théorème d'Hartman-Grobman, par exemple, est vrai en toute dimension $n \geq 2$, 
et la classification des systèmes linéaires est semblable. 

Considérons une matrice réelle $A$ de dimension quelconque. 
Si toutes ses valeurs propres sont distinctes 
alors on peut la diagonaliser par blocs réels $1 \times 1$, 
qui contiennent une valeur propre réelle, 
ou $2 \times 2$, de la forme $$\begin{pmatrix}
\alpha& \beta\\
-\beta& \alpha
\end{pmatrix},$$
 qui encodent une paire de valeurs propres conjuguées $\alpha \pm \beta i$.
  Dans ce cas le système linéaire (ou linéarisé)
   peut se décrire comme produit direct\footnote{Une union de systèmes découplés} de système uni-dimensionnels 
   ou bi-dimensionnels comme vus dans ce chapitre.



Brièvement, le cas de blocs de Jordan est un peu différent et se comporte comme suit. Un bloc de Jordan réel, par exemple 

$$\begin{pmatrix}
\lambda& 1 & \\
       & \lambda & 1 \\
       &         & \lambda
\end{pmatrix}$$ 
engendrera une dynamique fort semblable au cas bidimensionnel, combinaison linéaires de $e^{\lambda t}$, $te^{\lambda t}$, $t^2e^{\lambda t}$:

\eqnn
z_1(t)&=&z_1(0)e^{\lambda t}+ t z_2(0)e^{\lambda t}+ t^2 z_3(0)e^{\lambda t},\\
z_2(t)&=&z_2(0)e^{\lambda t}+ t z_3(0)e^{\lambda t},\\
z_3(t)&=&z_3(0)e^{\lambda t}.
\eeqnn

Un bloc de Jordan complexe se combine avec son conjugué pour former un bloc qui ressemble par exemple à ceci:

\eqn
\begin{pmatrix}
\alpha & \beta & 1 & 0 \\
-\beta & \alpha&  0 & 1 \\
   &      &  \alpha & \beta\\
   &      &   -\beta & \alpha \\
\end{pmatrix}
\eeqn
Les solutions ressemblent alors à ceci:
\eqnn
z_1(t)&=&e^{\alpha t}(z_1(0)\cos \beta t+z_2(0) \sin \beta t+ t(z_3(0)\cos \beta t+z_4(0) \sin \beta t)),\\
z_2(t)&=&e^{\alpha t} (z_2(0)\cos \beta t-z_1(0) \sin \beta t+t(z_4(0)\cos \beta t-z_3(0) \sin \beta t)),\\
z_3(t)&=&e^{\alpha t}(z_3(0)\cos \beta t+z_4(0) \sin \beta t),\\
z_4(t)&=&e^{\alpha t}(z_4(0)\cos \beta t-z_3(0) \sin \beta t).
\eeqnn



Illustrons maintenant les sections précédentes par quelques exemples de systèmes non
linéaires d'ordre 2.



\subsection{ Les systèmes mécaniques à un degré de
liberté}

Les {é}quations d'{é}tat d'un syst{è}me m{é}canique {à} un degr{é} de libert{é}
 s'{é}crivent (voir chapitre 2)~:
\begin{align*}
\dot x_1&=x_2,\\
m\dot x_2&=-g(x_1) -k(x_1)-h(x_2) + u,
\end{align*}
o{ù} $x_1$ est la coordonn{é}e de position du corps en mouvement, $x_2$ est la vitesse, $m$ d{é}signe la masse ou l'inertie et $u$ repr{é}sente une force ou un couple 
ext{é}rieur appliqu{é} au syst{è}me. Les fonctions scalaires $g(x_1)$ et $k(x_1)$ 
correspondent respectivement {à} la gravit{é} et
{à} l'élasticité tandis que $h(x_2)$ (tel que $h(0)=0$) repr{é}sente le frottement
visqueux. Le frottement sec est n{é}glig{é}. Notons aussi (voir chapitre~2, section~2.7) que 
\eqnn
g(x_1)+k(x_1) = \frac{\partial E_p}{\partial x_1}
\eeqnn
o{ù} $E_p$ d{é}signe l'{é}nergie potentielle du syst{è}me. 

Les {é}quilibres de ce syst{è}me sont caract{é}ris{é}s par
\begin{align*}
\bar x_2 &= 0,\\
g(\bar x_1)+k(\bar x_1) &= \bar u.
\end{align*}
Sans perte de généralité, considérons le cas particulier où $m=1$. La matrice Jacobienne du syst{è}me {à} l'{é}quilibre $(\bar x_1, 0,\bar
u)$ s'{é}crit~: 
\eqnn
 A=\bma{cc} 0 & 1\\-\left (\frac{\partial^2E_p}{\partial x_1^2}\right
)_{\bar x_1} & -h'(0) \ema.
\eeqnn
Le polyn{\^o}me caract{é}ristique de cette matrice est
\eqnn
p(x)=x^2+h'(0)x+\left (\frac{\partial^2E_p}{\partial x_1^2}\right )_{\bar
x_1}.
\eeqnn
 Le produit et la somme des valeurs propres sont donn{é}s par
$$ \lambda_1\lambda_2=\left (\frac{\partial^2E_p}{\partial x_1^2}\right
)_{\bar x_1}, \;\;\; \lambda_1+\lambda_2=-h'(0).$$
La d{é}riv{é}e $h'(0)$ du frottement visqueux est 
par nature non-n{é}gative~: $\lambda_1+\lambda_2 =-h'(0) \leq 0$.




Les {é}quilibres du syst{è}me sont hyperboliques si
 \eqnn 
h'(0) >0 &\mbox{et }&  \left (\frac{\partial^2E_p}{\partial x_1^2}\right
)_{\bar x_1} \neq 0,\\
 \mbox{ou si }  h'(0) =0 &\mbox{et }& \left (\frac{\partial^2E_p}{\partial x_1^2}\right
)_{\bar x_1} < 0.
\eeqnn
On observe donc que les {é}quilibres ne sont {\em pas} hyperboliques si l'{é}nergie potentielle $E_p(x_1)$ est une fonction affine de la position $x_1$, ou plus g{é}n{é}ralement si l'{é}quilibre correspond {à} un point d'inflexion de $E_p(x_1)$. C'est {é}galement le cas lorsque  $h'(0)=0$ et $\left (\frac{\partial^2E_p}{\partial
x_1^2}\right )_{\bar x_1} \geq 0$.

Les {é}quilibres hyperboliques d'un syst{è}me m{é}canique {à} un degr{é} 
de libert{é} peuvent alors {ê}tre compl{è}tement caract{é}ris{é}s 
comme indiqu{é} au tableau~\ref{tab:tabsysmec} (voir aussi la figure~\ref{fig:eqsysmec}).
\begin{figure}[t] 
   \centering
   \includegraphics[height=4cm]{eqsysmec} 
   \caption{Lieu des valeurs propres des équilibres d'un syst{è}me m{é}canique {à} un degr{é} de libert{é}}
   \label{fig:eqsysmec}
\end{figure}
\begin{table}
\hspace*{5mm}
%\renewcommand{\arraystretch}{1.2}
\begin{tabular}{|c|c|}
\hline
Caract{é}risation&Nature des {é}quilibres hyperboliques\\ \hline
{}&{}\\
$0< [h'(0)]^2 <4 \left (\frts{\partial^2E_p}{\partial x_1^2}\right
)_{\bar x_1}$&foyer stable \\
{}&{}\\ \hline
{}&{}\\
$0<4 \left (\frts{\partial^2E_p}{\partial x_1^2}\right
)_{\bar x_1}\leq [h'(0)]^2$&noeud stable \\
{}&{}\\
 \hline
 {}&{}\\
$\left (\frts{\partial^2E_p}{\partial x_1^2}\right
)_{\bar x_1} <0$&col \\ 
{}&{}\\
\hline
\end{tabular}
\caption{Equilibres hyperboliques des syst{è}mes m{é}caniques à un degré
de liberté}
\label{tab:tabsysmec}
\end{table}
On observe en particulier qu'un {é}quilibre hyperbolique ne peut jamais {ê}tre un noeud ou un foyer répulsif.

\subsection{Les circuits électriques RLC}

Les circuits {é}lectriques simples qui ne contiennent qu'une inductance et 
une capacit{é} sont g{é}n{é}ralement d{é}nomm{é}s {\em circuits 
RLC} dans la litt{é}rature. Dans les ouvrages de r{é}f{é}rence en 
g{é}nie {é}lectrique ou en th{é}orie des circuits, ils font l'objet 
d'une {é}tude approfondie car ils constituent la configuration de base de
 nombreux dispositifs pratiques (filtres, oscillateurs,...).

Le circuit {\em RLC s{é}rie} repr{é}sent{é} {à} la figure~\ref{fig:RLCs}
est un exemple typique.
\begin{figure}[htbp] 
   \centering
   \includegraphics[height=2.5cm]{RLCs} 
   \caption{Circuit RLC s{é}rie}
   \label{fig:RLCs}
\end{figure}
En application des principes {é}tudi{é}s au chapitre~3, le 
comportement dynamique de ce circuit est d{é}crit par un mod{è}le 
d'{é}tat de dimension 2~:
\eqnn
L \dot x_1&=&-r(x_1)-x_2 +u\\
C \dot x_2&=&x_1
\eeqnn
o{ù} $x_1=i$ est le courant dans l'inductance linéaire $L$, $x_2=v$ est 
la tension aux bornes de la capacit{é} linéaire $C$ et $r(x_1)$ est la 
caract{é}ristique tension-courant ({é}ventuellement non lin{é}aire)
de la r{é}sistance.
\begin{figure}[t] 
   \centering
   \includegraphics[height=4cm]{figrlc} 
   \caption{Lieu des valeurs propres des {é}quilibres d'un circuit RLC}
   \label{fig:figrlc}
\end{figure}

Les {é}quilibres de ce syst{è}me sont caract{é}ris{é}s par 
les {é}quations~:
\begin{align*}
\bar x_2 +r(0) &= \bar u,\\
\bar x_1 &= 0.
\end{align*}
Sans perte de généralité, considérons le cas particulier $L=1$ et $C=1$. La matrice Jacobienne du syst{è}me {à} l'{é}quilibre $(0,\bar x_2,\bar u)$ s'{é}crit~:
$$A=\bma{cc} -r'(0) & -1\\1&0\ema .$$
Le polyn{\^o}me caract{é}ristique de cette matrice est~:
\eqnn
p(x)&=&\lambda^2+r'(0) \lambda +1\\
\mbox{o{ù} }\;\; r'(0)&\triangleq&(\partial r/\partial x_1)_{x_1=0}.
\eeqnn
Le produit et la somme des valeurs propres sont donn{é}s par
$$ \lambda_1\lambda_2=1,  \;\;\; \lambda_1+\lambda_2=-r'(0).$$
Les {é}quilibres du syst{è}me sont donc hyperboliques si $r'(0)\neq 0$, 
c.{à}.d. si la d{é}riv{é}e de la caract{é}ristique de la r{é}sistance n'est pas 
nulle {à} l'origine. On observe que c'est notamment le cas pour une 
r{é}sistance lin{é}aire.

Les {é}quilibres hyperboliques d'un circuit RLC s{é}rie sont alors compl{è}tement
caract{é}ris{é}s comme indiqu{é} sur le tableau \ref{tabrlc} 
(voir aussi la figure~\ref{fig:figrlc}).
\begin{table}
\centering
\renewcommand{\arraystretch}{3.0}
\begin{tabular}{|c|c|}
\hline
&Nature des {é}quilibres hyperboliques\\ \hline
$r^{'}(0) \geq 2$&noeud attractif \\ \hline
$0 < r^{'}(0) < 2$&foyer attractif \\ \hline
$-2 < r^{'}(0) < 0$&foyer répulsif \\ \hline
$r^{'}(0) \leq -2 $& noeud répulsif\\
\hline
\end{tabular}
\caption{Equilibres hyperboliques d'un circuit RLC}\label{tabrlc}
\end{table}
On remarque en particulier qu'un {é}quilibre hyperbolique d'un circuit RLC s{é}rie ne peut jamais {ê}tre un col.

\subsection{Les systèmes à deux compartiments}

Consid{é}rons les syst{è}mes {à} deux compartiments dont le graphe est repr{é}sent{é} {à} la
figure~\ref{fig:deuxcomp}.
\begin{figure}[htbp] 
   \centering
   \includegraphics[height=4cm]{deuxcomp} 
   \caption{Système à deux compartiments}
   \label{fig:deuxcomp}
\end{figure}
Le signal d'entr{é}e $u$ est le d{é}bit d'alimentation du premier compartiment.  Nous
supposons que les flux {é}chang{é}s entre les compartiments satisfont les conditions
$C1 - C4$ de mod{é}lisation du chapitre 4 (Section 4.3).  La dynamique du syst{è}me
est alors d{é}crite par un mod{è}le d'{é}tat de dimension 2 de la forme g{é}n{é}rale
suivante :
\eqnn
\dot x_1 &=& -q_{12} (x_1, x_2) + q_{21} (x_2,x_1) - q_{10} (x_1) + u\\
\dot x_2 &=& q_{12}(x_1,x_2)-q_{21} (x_2,x_1) - q_{20} (x_2)
\eeqnn
Les fonctions $q_{ij}$ satisfont les conditions suivantes
sur l'orthant positif :
\e
q_{ij}(0, x_j) = 0 \;\;\; \frac{\partial q_{ij}}{\partial
x_{i}} \geq 0 \;\; \frac{\partial q_{ij}}{\partial
x_{j}} \leq 0 \label{orthpositif}
\ee
Sous ces conditions, le
syst{è}me poss{è}de une infinit{é} d'{é}quilibres isol{é}s positifs
$(\bar x_1, \bar x_2, \bar u)$.  La matrice Jacobienne
autour de l'un de ces {é}quilibres s'{é}crit :
$$
A = \bma{cc} -(a+c) &  b\\a & -(b+d) 
\ema
$$
avec les notations simplifi{é}es suivantes (toutes les
d{é}riv{é}es partielles sont {é}valu{é}es {à} l'{é}tat d'{é}quilibre ) :
\eqnn
a &\triangleq & \frac{\partial q_{12}}{\partial x_1} - \frac{\partial
q_{21}}{\partial x_1} \hspace*{15mm}c \triangleq \frac{\partial q_{10}}{\partial x_1}\\
b &\triangleq & \frac{\partial q_{21}}{\partial x_2} - \frac{\partial
q_{12}}{\partial x_2} \hspace*{15mm} d \triangleq \frac{\partial q_{20}}{\partial x_2}
\eeqnn
Sous les conditions (\ref{orthpositif}), on observe imm{é}diatement que
$a,b,c,d, \geq~0$.  Le polyn{\^o}me caract{é}ristique de la matrice Jacobienne s'{é}crit :
$$
p(x) = x^2 + (a+b+c+d)x + (ad+bc+cd)
$$
Le produit et la somme des valeurs propres sont donc donn{é}s par :
$$
\lambda_1 \lambda_2 = ad+bc+cd \;\;\;\; \lambda_1 + \lambda_2 = -(a+b+c+d)
$$
Les {é}quilibres du syst{è}me sont donc hyperboliques si les in{é}galit{é}s suivantes
sont satisfaites :
$$
a+b+c+d > 0 \;\;\mbox{ et }\;\; ad+bc+cd > 0
$$
On d{é}montre ais{é}ment que, sous ces conditions, l'in{é}galit{é} suivante est aussi
satisfaite :
$$
0<4 (ad+bc+cd) \leq (a+b+c+d)^2
$$
On en d{é}duit que les {é}quilibres hyperboliques d'un syst{è}me {à} deux
compartiments ne peuvent {ê}tre que des noeuds attractifs (voir
figure~\ref{fig:eq2comp}).
\begin{figure}[htbp] 
   \centering
   \includegraphics[height=4cm]{eq2comp} 
   \caption{Lieu des valeurs propres des {é}quilibres d'un syst{è}me {à} 
deux compartiments}
   \label{fig:eq2comp}
\end{figure}

\subsection{Les systèmes réactionnels à deux espèces}

Les syst{è}mes r{é}actionnels les plus simples font intervenir deux esp{è}ces. 
C'est le cas par exemple d'une r{é}action irr{é}versible convertissant un r{é}actif
$X_1$ en un produit $X_2$ :
$$
X_1 \longrightarrow X_2
$$
Supposons que cette r{é}action se d{é}roule dans un r{é}acteur continu parfaitement
m{é}lang{é} {à} volume constant.  Le r{é}acteur est aliment{é} avec l'esp{è}ce
 $X_1$, {à}
d{é}bit volum{é}trique constant strictement positif.  Comme nous l'avons vu au
chapitre 5, le mod{è}le d'état du r{é}acteur peut s'{é}crire comme suit :
\eqnn
\dot x_1 &=& -r(x_1,x_2) + d(u-x_1)\\
\dot x_2 &=& r(x_1,x_2) - dx_2
\eeqnn
o{ù}  $x_1$ et $x_2$ représentent les concentrations des esp{è}ces $X_1$ et $X_2$ dans le
milieu r{é}actionnel, $d$ est le taux de dilution et $u$ est la concentration du
r{é}actif $X_1$ dans l'alimentation.  La cin{é}tique de r{é}action $r(x_1,x_2)$ est
suppos{é}e {ê}tre une fonction des concentrations des deux esp{è}ces.
\begin{figure}[t] 
   \centering
   \includegraphics[height=4cm]{fig2esp} 
   \caption{Lieu des valeurs propres des {é}quilibres d'un syst{è}me
 r{é}actionnel {à} deux esp{è}ces}
   \label{fig:fig2esp}
\end{figure}

Les {é}quilibres du syst{è}me sont donc caract{é}ris{é}s par les {é}quations  :
$$
d\bar x_2 = d(\bar u-\bar x_1) = r(\bar x_1, \bar x_2)
$$
Ces {é}quations impliquent {à} l'{é}quilibre que $\bar x_1 + \bar x_2 = \bar u$,
c'est {à} dire que la somme des concentrations des esp{è}ces $X_1$ et $X_2$ dans le
r{é}acteur est {é}gale {à} la concentration du r{é}actif $X_1$ dans l'alimentation. 
Cette observation est {é}videmment en accord avec le principe de conservation de
la masse.

La matrice Jacobienne autour de l'{é}quilibre s'{é}crit :
$$
A = \bma{cc}
-a-d & -b \\a & b-d
\ema
$$
avec les notations simplifi{é}es suivantes :
$$
a  \triangleq \left( \frac{\partial r(x_1,x_2)}{\partial x_1} \right )_{\bar
x_1, \bar x_2} \hspace*{15mm} b \triangleq \left( \frac{\partial r(x_1,x_2)}{\partial x_2} \right )_{\bar
x_1, \bar x_2} 
$$
Le polyn{\^o}me caract{é}ristique de la matrice Jacobienne s'{é}crit :
$$
p(x) = x^2 + (a-b +2d) x + (a-b) d +d^2
$$
Le produit et la somme des valeurs propres sont donc donn{é}s par :
$$
\lambda_1 \lambda_2 = (a-b)d+d^2 \;\;\;\; \lambda_1 + \lambda_2 = -(a-b+2d)
$$
Etant donn{é} que le taux de dilution $d$ est une quantit{é} strictement positive,
on peut v{é}rifier apr{è}s quelques calculs que les {é}quilibres du syst{è}me sont
hyperboliques si $(a-b) \neq -d$. On observe que 
\begin{itemize}
\item si $\lambda_1 + \lambda_2 = -[(a-b)+2d]>0$, alors n{é}cessairement
$\lambda_1 \lambda_2 = d[(a-b)+d]<0$ et donc l'{é}quilibre est un col.
\item Si $\lambda_1 + \lambda_2 = -[(a-b)+2d]\leq 0$, alors l'{é}quilibre est un col
si $-2d\leq(a-b)<-d$, et un noeud attractif si $(a-b)>-d$.  Par contre,
l'{é}quilibre ne peut pas {ê}tre un foyer, car il est impossible d'avoir
$\lambda_1 \lambda_2 \geq \frac{1}{4} (\lambda_1 + \lambda_2)^2$.
\end{itemize}
Cette analyse est r{é}sum{é}e dans 
le tableau~\ref{tab2esp} et la figure~\ref{fig:fig2esp}.

\begin{table}
\hspace*{10mm}
\renewcommand{\arraystretch}{3.0}
\begin{tabular}{|c|c|}
\hline
&Nature des {é}quilibres hyperboliques\\ \hline
$(a-b)<-d$&col \\ \hline
$(a-b)>-d$&noeud attractif \\ \hline
\end{tabular}
\caption{Equilibres hyperboliques d'un syst{è}me r{é}actionnel {à} deux esp{è}ces}
\label{tab2esp}
\end{table}

\section{Periodic trajectories and limit cycles}

From the tables of the sections~8.2, we can make the following observations.
\begin{enumerate}
\item For a linear system of dimension 2, the {\em attractive} equilibriums either are a node or a spiral or a line of uninsulated equilibriums. in
each of these cases, the basin of attraction is the entire phase map.
\item When the equilibrium is repulsive, the system trajectories diverge when the time $t$ tend to infinity.
\item When the equilibrium of a linear system is a center, all the system trajectories are periodic and the radius pf the trajectories depend on the initial conditions.
A linear system with periodi trajectories is structurally unstable, and thus any perturbation of the system can remove these periodic trajectories.
\end{enumerate}
None of this observations is generically verified in the case of nonlinear systems. Indeed, the first two concern a {\em global} behaviour of the trajectories, and we have seen that this is only locally, in the neighborhood of a hyperbolic equilibrium, that
trajectories of a nonlinear system behave like those of the linear approximation of the system.

The purpose of this section is to show that, for non linear systems, there are other attractive sets, and in particular periodic trajectories.
Futhermore, we will show that these attractive sets are structurally stable. This is a very interesting property of the linear systems wich is used for the conception of oscillator circuits.

\begin{exemple} {\bf  \em Diode tunnel RLC circuits} 

\begin{figure}[htbp] 
   \centering
   \includegraphics[height=25mm]{osc} 
   \caption{Tunnel diode oscillator}
   \label{fig:osc}
\end{figure}

The figure~\ref{fig:osc} shows a tunnel diode oscilator. This is an electric circuit with linear dipoles (a constant voltage source $E$, a variable linear resistance $R$, a linear inductance $L= 1H$, a linear capacitor $C=1F$) and a non linear resistance (tunnel diode)
whose current-voltage characteristic $i=h(v)=2v^3-6v^2+5v$ has the shape of the curve shown in figure~\ref{fig:diode}. The input of this system is the variable resistence $R$.

\begin{figure}[htbp] 
   \centering
   \includegraphics[width=10cm]{caradio} 
   \caption{Tunnel diode current-voltage characteristic.}
   \label{fig:caradio}
\end{figure}
As we have seen in chapter~3, the state variables of the system are the current $x_1=i$ in the inductance and the voltage $x_2=v$ at the terminals of the capacitor.
One obtains the following state equations~:

\eqnn
\dot x_1&=&-u x_1-x_2 +E\\
\dot x_2&=&x_1-h(x_2),
\eeqnn
and the possible equilibriums are characterized by
\eqnn
\bar x_1&=&\frac{E -\bar x_2}{\bar u}\\
\bar x_1&=&h(\bar x_2).
\eeqnn

By showing in the phase plane the graphs of the curves $
\bar x_1=(E -\bar x_2)/\bar u$ et $\bar x_1=h(\bar x_2)$, one notes that, for a given characteristic of diode,
two configurations are possible according to the respective values of

$\bar u$ and $E$. If the line ($-1/\bar u$) is steep enough, there will be only one equilibrium point
(figure~\ref{fig:eqdiode}.a).  On the other side, if this slope is below the one of the the tangent at the inflection point of the curve, there will be
one, two or three possible equilibriums according to the value of $E$.
(figure~\ref{fig:eqdiode}.b).
\begin{figure}[htbp] 
   \centering
   \includegraphics[width=10cm]{eqdiode} 
   \caption{Equilibrium configurations for the tunnel diode circuit}
   \label{fig:eqdiode}
\end{figure}

We can study again the shape of the trajectories at the neighborhood of the equilibrium by calculating the eigenvalues of the Jacobian matrix of the system~:
$$A=\bma{cc} -\bar u & -1\\1&-h'(\bar x_2)\ema .$$

The product and the sum of the eigenvalues are given by
$$ \lambda_1\lambda_2=\bar u h'(\bar x_2) +1,  \;\;\; \lambda_1+\lambda_2=-(\bar u+
h'(\bar x_2)),$$

and one observes that the sign of the eigenvalues doesn't depend on $E$ but only depends on the respective slopes of the two graphs of either of figures~\ref{fig:eqdiode}.

Let's examine in details the equilibriums~:

\begin{itemize}

\item[{\bf a.}] For the figure~\ref{fig:eqdiode}.a, there is only one equilibrium. If this one 
is to the left of the local maximum of the curve $h(x_2)$ or to the right of the local minimum of this one, the product
of the eigenvalues is positive, and thus the corresponding equilibrium is a node or a stable spiral.

\item[{\bf b.} ] Always for the first figure, if the equilibrium stand between the local maximum and the local minimum, 
we have $-1/\bar u < h'(\bar x_2)<0$ and the product of the eigenvalues
is thus positive.
Regarding the sum, it will be negative and the corresponding equilibrium
will therefore be attractive if $|h'(\bar x_2)|<\bar u$ (wich corresponds to an important value of $\bar u$ ie a 
a strongly dissipative resistance which ensures the stability of the circuit). On the other side, if $|h'(\bar x_2)|>\bar u$,
the sum of the eigenvalues is positive and the corresponding equibibrium is repulsive.

\item[{\bf c.}] Regarding the figure~\ref{fig:eqdiode}.b,, the equilibriums to the left of the local maximum of 
$h(x_2)$ and to the left of the local minimum are such that the product of the eigenvalues is positive and the sum of the eigenvalues
is negative. The corresponding equilibrium is either a node or a stable spiral.

\item[{\bf d.}] Regarding the eventual equilibrium between the maximum and the minium, it is it verifies $h'(\bar x_2) < -1/\bar u < 0$. The product of the
eigenvalues is negative and the corresponding equilibrium is a saddle.

\end{itemize}

As we can note, the equilibrium is repuslive in different cases. We can then question about what the trajectories moving away from this equilibrium point become.

Let's consider the following particular numerical values~:

\eqnn
\bar u&=&0.5,\\
E&=& 1.5,\\
h(v)&=&2v^3-6v^2+5v.
\eeqnn

We can check that for these particular values, $(\bar x_1, \bar x_2, \bar u)= (1,1,0.5)$ is the only equilibrium of the system.
and that it is a repulsive equilibrium (case {\bf b.} above).

By simulating the system of two differential equations for different initial conditions, we obtain the orbits illustrated in figure~\ref{fig:diode}. 
It clarly appears that all the calculated orbits (we could think that others wound behave the same way wrap around a periodical orbit.
This system has no attractive equilibrium, but there exists a {\em closed orbit} wich is attractive.
This is what is called a limit cycle.


\begin{figure}[htbp] 
   \centering
   \includegraphics[width=10cm]{diode} 
   \caption{Limit cycle for a tunnel diode circuit}
   \label{fig:diode}
\end{figure}

Figure~\ref{fig:simudiodet} shows the trajectories (state as function of the time) and 
shows that they (quickly) converge toward periodic trajectories whose periode and amplitude don't depend on the initial conditions.

\begin{figure}[htbp] %  figure placement: here, top, bottom, or page
   \centering
   \includegraphics[width=10cm]{diodet} 
   \caption{Trajectories of the tunnel diode circuit}
   \label{fig:simudiodet}
\end{figure}

asymptotically, therefore, the system will undergo amplitude oscillations
constant, whatever the value of the initial conditions, contrary to what happens for a linear system with a center type of equilibrium.

In fact, this is exactly what is sought when building an oscillator: Oscillations of constant amplitude 
regardless of initial conditions, we can not get that with a non-linear system. Finally, it is
also show that the limit cycle is structurally stable, which is also
an interesting property for the design of an oscillator.
\qed
\end{exemple}

We formalize below some of the concepts that have just been
described in the preceding example. Consider a plannar system $$\dot x=f(x,\bar u)$$ 
with constant input $\bar u$ and let's note $x(t,x_0,\bar u)$ the solution at time $t$ with $x(0)=x_0$.

\begin{definition}{\bf\em Limit point}

The point $z$ is a {\em limit point of $y$} for the dynamical system submitted to a constant input $\bar u$
if there exists a sequence $\{t_n\}$ in
$\RR$ such that $t_n \rightarrow \infty$ when $n \rightarrow \infty$ and
$\lim_{n\rightarrow \infty} x(t_n,y,\bar u)=z$.\qed
\end{definition}

According to this definition, a atttractif equibrium is therefore an endpoint
from any point in its basin of attraction. But the notion of limit point is more
general as we will see below.

\begin{definition}{\bf \em Limit cycle}

{\em A {\em limit cycle} is a closed orbit $\gamma$ such that a point of $\gamma$ is a limit point of another pojnt of the phase plane not in 
$\gamma$.}\cqfd\end{definition}

This definition shows that when closed orbit is a limit cycle, while
point of the orbit is a boundary point, and thus the trajectory of the system
will approach more and more each point of the closed orbit, to
instants determined.
We can now state some results to establish the existence
periodic trajectories and limit cycles. These results are only valid
for plannar systems (while there are also limitations cycles
systems of higher order). The reason is that these demonstrations
results are based on the fact that in two dimension, a closed orbit in the plane of
stage divides the plan into an inner region in the orbit and an outer region,
This is of course true in a phase space dimension greater than
2. The first result is a sufficient condition for non-existence path
periodic (and therefore limit cycle).

\begin{theoreme} {\bf \em Bendixson Dulac, 1901 et 1933} 

{\em Let $D$ be a simply connex domain \footnote{a connex domain in  $\RR^2$ whose frontier can be obtained by deformation of a circle.}
If Si $${\mbox div}f \triangleq \frac{\partial f_1}{\partial
x_1}+\frac{\partial f_2}{\partial x_2}$$ is non identically zero in a subdomain of $D$ and does not change sign in this subdomain,
then $D$ does not contain any closed orbit.}\cqfd\end{theoreme}

Let's recall that that the divergence ${\mbox div}f$ describes the remoteness (${\mbox div}f >0$) or the rapprochement (${\mbox div}f <0$) from $\dot{x}=f(x)$.
This theorem is proved simply by contradiction: suppose that there exists a closed path $\gamma$ in the domain, whose interior is $D_{\gamma} \subseteq D$.
Thus the integral of the divergence inside of $\gamma$, $\int_{D_\gamma}  {\mbox div}f$ is equal by the Green-Stokes theorm to the integral
of the flux through the frontier $\gamma$ $\int_\gamma \langle f, n \rangle$, where $\langle.,.\rangle$
is the scalar product and $n$ is the unitary normal vector exiting from $\gamma$.
This integral is zero, since $f$ is tangent everywhere in every point of the trajectory $\gamma$. 

It follows that the divergence ${\mbox div}f$ can not be negative everywhere or positive everywhere in $D_{\gamma}$.
The second result allows to bring out the existence of a limit cycle.

\begin{theoreme} {\bf \em Poincar{é}-Bendixson, 1901}

If $E$ is a closed and bounded subset of $\RR^2$, invariant for the system $\dot x= f(x,\bar u)$, and if $\gamma$ is an orbit
starting in $E$, then~:

\begin{itemize}
\item[i)] If $E* doen't contain any equilibrium point, then $\gamma$ either is a periodic orbit or converge toward a limit cycle.

\item[ii)] If $E$ doesn't contain any periodic orbit but contains a unique equilibrium point, this equilibrium is globally attractive in $E$.\qed

\end{itemize}
\end{theoreme}


Ensuite, si on a pu exclure la
pr{é}sence d'{é}quilibres dans cet ensemble, celui-ci doit n{é}cessairement contenir un
cycle limite, ou ne contenir que des trajectoires p{é}riodiques.

This theorem can be used to prove the existence of a limit cycle. To do this, we first seek a closed, bounded and invariant set.
To check that the set is invariant, we show that on the border of this set, de vector field points inwards. Then, if we have been 
able to exclude an equilibrium in this set, this one must necessarily contain a limit cycle, or contain only periodic trajectories.

It is important to notice that this two theorems, unlike other results of this chapter, are specific to the plannar systems, 
and strongly restrict the possible dynamics in two dimensions : the trajectories converge to a point, a cycle, or are unbounded.
The higher dimensions have a richer behaviour wich exceed the purpose of this course : chaos, strange attractors, etc.

\begin{exemple}{\bf \em Circuit {à} diode tunnel (suite)}

\begin{exemple}{\bf \em Tunnel diode circuit}

We consider again the circuit already described with the same numerical values than previously, which lead to a unique 
repulsive equilibrium $(\bar x_1, \bar
x_2, \bar u)=(1,1,E-1)$, with $E>1$. Let's now consider in the plannar phase a circle centred in $(0,0)$ and with a large enough
radius, and let's show that, on the circle, the vector field points inwards.

It is therefore to show that the scalar product of the vector field and the normal vector of the circle is negative ~: $PS=x_1f_1(x_1,x_2,\bar
u)+x_2f_2(x_1,x_2,\bar u)  < 0$. Let's choose as a radius $r=\sqrt{2}\frac{E}{E-1}$ (see figure~\ref{bendiode}).

\begin{figure}[htbp] 
   \centering
   \includegraphics[width=10cm]{bendiode} 
   \caption{Invariant set for the tunnel diode circuit}
   \label{bendiode}
\end{figure}

The scalar product is equal to $PS=-(E-1) x_1^2+E x_1-x_2h(x_2)$. Note that the quantity $-x_2h(x_2)$ is always nonpositive except for $x_2=0$.
\leq 0, PS < 0$. Likewise for $x_1 \geq \frac{E}{E-1}, E x_1 \leq (E-1)x_1^2$ and $PS
<0$. 

Its remains to study the portion of the circule where $x_1 < \frac{E}{E-1}, x_2 > \frac{E}{E-1}.$ A small calculation allows
to check that $|h(x_2)| > |x_2|$ and that the following inequalities are thus checked~:

\eqnn x_2h(x_2)&>x_2^2&>\frac{E^2}{(E-1)^2}\\
E x_1&<\frac{E^2}{E-1}&<\frac{E^2}{(E-1)^2}
\eeqnn

and then $PS < 0$. On the circule of radius $r$, the vector field points thus inwards. Otherwise, given that the equilibrium
$(1,1,E-1)$ is repulsive, we can take a small enough cicrule around this equilibrium such that the vector field evaluated
on the circle points outwards. If we now consider that the domain formed by the ring (not centered) between the small circle 
and the big one, it is an invariant set because on the border of this set, the vector field points inward the domain.
Given that the domain doesn't inclure any équilibrium, it must contain a limit cycle (or contain periodic trajectories only). \qed
\end{exemple}

%Notons encore qu'il existe d'autres ensembles attractifs que des {é}quilibres %ou des
%cycles limites. Un syst{è}me comprenant trois {é}quilibres dont deux sont des %foyers
%instables et le troisi{è}me un col peut avoir un portrait de phase comme %illustr{é} {à} la
%figure~\ref{fig:syshuit}. Dans cet exemple, le ``huit'' illustr{é} sur la figure %est
%l'ensemble attractif pour tous les points {à} l'ext{é}rieur de cette courbe %ferm{é}e,
%alors que chacune de ses branches est l'ensemble attractif des points {à} %l'int{é}rieur.
%Les branches du ``huit'' sont des orbites de trajectoires tendant vers le col %(et ne sont
%donc pas des orbites p{é}riodiques).

%\begin{figure}
%\vspace*{6cm}
%\caption{ensemble attractif complexe}\label{fig:syshuit}
%\end{figure}

\section{Bifurcations}

Nous avons choisi d'{é}tudier dans ce chapitre l'allure des trajectoires de syst{è}mes plans pour une valeur constante de l'entr{é}e, $\bar u$. Cette valeur n'{é}tant pas
n{é}cessairement fix{é}e {\em a priori}, il est int{é}ressant d'analyser dans quelle
mesure les trajectoires seront influenc{é}es par des changements de $\bar u$. Le
th{é}or{è}me~\ref{Hart} nous donne d{é}j{à} une indication. Tant que l'{é}quilibre autour
duquel on analyse les trajectoires est hyperbolique, de petites variations de $\bar u$ ne
d{é}placeront pas beaucoup les valeurs propres de la matrice d'{é}tat de l'approximation
lin{é}aire du syst{è}me, et l'allure des trajectoires restera similaire. Mais en faisant
varier l'entr{é}e constante $\bar u$, il peut arriver que les
valeurs propres de la matrice d'{é}tat atteignent l'axe imaginaire du plan complexe, et
dans ce cas il faut s'attendre
{à} une modification fondamentale de l'allure des trajectoires. Plus globalement, les
diagrammes d'{é}quilibre {é}tudi{é}s au chapitre pr{é}c{é}dent montrent {é}galement qu'en
faisant varier $\bar u$, on peut modifier le nombre de points d'{é}quilibre du syst{è}me,
autant que leur nature. L'{é}tude des modifications de la nature et/ou du nombre des
{é}quilibres en fonction de l'{é}volution de l'entr{é}e du syst{è}me
rel{è}ve de ce qu'on appelle la th{é}orie des bifurcations, et l'entr{é}e constante $\bar u$ est alors
appel{é}e {\em param{è}tre de bifurcation}. Nous illustrons ci-dessous ce concept en
pr{é}sentant quatre types de bifurcations qui se rencontrent dans les syst{è}mes plans.

\subsection{Bifurcation de Hopf}

\begin{exemple} {\em\bf Circuit {à} diode tunnel (suite)}

Reprenons à nouveau l'exemple du circuit à diode tunnel en faisant varier l'entr{é}e 
$\bar u$ (c.{à}.d. la r{é}sistance variable $R$), avec une source de tension constante $E=1.5$.
La  figure~\ref{fig:eqdiohop} illustre comment l'{é}quilibre unique se d{é}place lorsque $\bar u$ varie. Le tableau suivant caract{é}rise le type d'{é}quilibre rencontr{é} en fonction de $\bar u$.
\begin{figure}[htbp] 
\centering
\includegraphics[height=65mm]{eqdiohop} 
\caption{Equilibre du circuit à diode tunnel lorsque la résistance $R$ varie.}
\label{fig:eqdiohop}
\end{figure}
\begin{table}
\begin{tabular}{|c|c|c|c|c|}\hline
$\bar u$&$\bar x_2$&$h'(\bar x_2)$&valeurs&type\\ 
&&&propres&d'{é}quilibre\\ \hline
$\bar u >0.7139$&$\bar x_2 < 0.5918$&$h'(\bar x_2)>0$&$\lambda_{1,2} \in C^-$&foyer\\
&&&&attractif\\ \hline
$0.1261<\bar u$&$0.5918< \bar x_2 $&$h'(\bar x_2)<0$&$\lambda_{1,2} \in  
C^+$&foyer\\ $<.7139$&$< 1.4082$&&&répulsif\\ \hline
$\bar u<0.1261$&$ \bar x_2 > 1.4082$&$2.5 >h'(\bar x_2)>0$&$\lambda_{1,2} \in
  C^-$&foyer\\
&&&&attractif\\ \hline
\end{tabular}
\end{table}
D{è}s lors, si l'on part d'une valeur de  la r{é}sistance variable $\bar u$ suffisamment grande, telle que le point d'{é}quilibre se
trouve {à} gauche du premier sommet de la courbe caract{é}ristique de la diode, et que
l'on diminue progressivement cette valeur, on passe successivement par les configurations suivantes~:  un foyer attractif, un foyer répulsif (associ{é} {à} un cycle limite), un foyer
attractif. Au moment des deux transitions entre foyer attractif et
répulsif, le syst{è}me passe par une valeur telle que le point d'{é}quilibre n'est pas
hyperbolique. 
\qed
\end{exemple}

La bifurcation que nous venons de mettre en {é}vidence
(passage d'un foyer attractif {à} un foyer répulsif accompagn{é} d'un cycle limite, ou l'inverse) est
appel{é}e {\em bifurcation de Hopf}. Le th{é}or{è}me suivant garantit d'ailleurs
l'existence d'un cycle limite. Afin de l'{é}noncer de fa\c con pr{é}cise, formalisons ce qui pr{é}c{è}de.  Soit un syst{è}me plan poss{é}dant une famille d'{é}quilibres uniques $(\bar x, \bar u)$ param{é}tr{é}e par $\bar u$. On suppose qu'il existe une valeur $\bar u^*$ de $\bar u$ 
telle que les valeurs propres de la matrice Jacobienne évaluée en cet
 {é}quilibre ont une partie r{é}elle nulle et une partie imaginaire non nulle. Ces valeurs propres d{é}pendent contin{\^u}ment de $\bar u$, au moins dans un voisinage de $\bar u^*$, et on les notera donc $$\lambda_i(\bar u)=\alpha(\bar u)\pm i \beta(\bar u).$$ On suppose en outre que $\frac{d\alpha(\bar u^*)}{d\bar u}>0$.

\begin{theoreme}
Avec les hypoth{è}ses qui pr{é}c{è}dent, si pour des valeurs de
 $\bar u$ proches de $\bar u^*$, l'{é}quilibre est attractif 
 pour $\bar u <\bar u^*$ et répulsif pour $\bar u >\bar u^*$ alors il
  existe une orbite ferm{é}e pour $\bar u>\bar u^*$ ou pour $\bar u<\bar u^*$. 
  En particulier, si $(\bar x^*,\bar u^*)$ est localement
   attractif, alors il existe un cycle limite attractif autour de 
  $(\bar x,\bar u)$ pout tout $\mu=\bar u-\bar u^*>0$, suffisamment petit.
   De plus, l'amplitude du cycle limite augmente lorsque $\mu$ augmente.
   \qed
\end{theoreme}
   
\begin{remarque}
Tel quel, l'{é}nonc{é} du th{é}or{è}me reste 
ambigu quant {à} la nature (attractive ou r{é}pulsive) de l'orbite 
ferm{é}e qui apparaît. On peut lever cette ambiguïté au prix 
d'un {é}nonc{é} plus technique faisant apparaître explicitement 
les termes d'ordre trois du syst{è}me non lin{é}aire
(voir par exemple Guckenheimer et Holmes, 
{\em Nonlinear Oscillations, Dynamical Systems, and Bifurcations of Vector 
Fields}, Springer-Verlag, 1983). 
\end{remarque}

\subsection{Bifurcation transcritique}

Consid{é}rons  la r{é}action $$X_1
+ X_2 \rightarrow 2 X_2$$ se produisant dans un r{é}acteur {à} volume constant, aliment{é}
en r{é}actif $X_1$ {à} la concentration $x_1^{in}$, avec un taux de dilution $u$.

Le mod{è}le d'état du syst{è}me (en supposant une cin{é}tique de r{é}action d{é}crite
par la loi d'action des masses) est donn{é} par
\eqnn
\dot x_1&=& -kx_1x_2+u(x_1^{in} -x_1)\\
\dot x_2&=& kx_1x_2-ux_2.
\eeqnn

Le syst{è}me poss{è}de deux {é}quilibres distincts pour chaque valeur constante de
l'entr{é}e $\bar u \neq kx_1^{in}$~: $(x_1^{in},0,\bar u)$ et $(\bar u/k,x_1^{in}-\bar
u/k,\bar u)$, comme illustr{é} {à} la figure~\ref{fig:diageqtc}.
\begin{figure}[htbp] 
   \centering
   \includegraphics[height=4.6cm]{diageqtc} 
   \caption{Diagramme d'{é}quilibres - Bifurcation transcritique}
   \label{fig:diageqtc}
\end{figure}
On v{é}rifie facilement que le premier {é}quilibre est attractif si $\bar u > kx_1^{in}$ et
est un col sinon. Inversement, le deuxi{è}me {é}quilibre est attractif pour les petites
valeurs de $\bar u$ et devient un col si  $\bar u > kx_1^{in}$. Il y a donc ici aussi une
bifurcation, plus simple toutefois, les caract{é}ristiques des deux {é}quilibres {é}tant
{é}chang{é}es lorsque le param{è}tre de bifurcation $\bar u$ franchit la valeur critique
$kx_1^{in}$. Cette bifurcation est appel{é}e {\em bifurcation transcritique}. On
v{é}rifie {é}galement qu'{à} cette valeur critique, l'{é}quilibre (unique) est non
hyperbolique.

\subsection{Bifurcation col-noeud}

Le troisi{è}me type de bifurcation est illustr{é} par l'exemple du r{é}acteur chimique
exothermique d{é}crit {à} la section~7.1. Rappelons que le diagramme d'{é}quilibre
reliant la temp{é}rature d'{é}quilibre du r{é}acteur, $\bar T$, {à} l'apport calorifique
externe, $\bar u$, a l'allure illustr{é}e {à} la figure~\ref{fig:diageqcn}.
\begin{figure}[htbp] 
   \centering
   \includegraphics[height=5cm]{diageqcn} 
   \caption{Diagramme d'{é}quilibres - Bifurcation col-noeud}
   \label{fig:diageqcn}
\end{figure}
On constate donc que pour de faibles valeurs de $\bar u$, le syst{è}me poss{è}de un seul
point d'{é}quilibre correspondant {à} une temp{é}rature d'{é}quilibre basse et {à} une
grande concentration de r{é}actif dans le r{é}acteur (et d{è}s lors une faible
concentration du produit de la r{é}action). On peut v{é}rifier que cet {é}quilibre est
attractif. Puis, pour une valeur critique de $\bar u$ que l'on rep{è}re facilement sur le
diagramme d'{é}quilibre, le syst{è}me passe {à} trois valeurs d'{é}quilibre pour la
temp{é}rature, celle du milieu correspondant {à} un {é}quilibre attractif et les deux autres
{à} des {é}quilibres répulsifs. Enfin, en augmentant encore $\bar u$, on franchit une
nouvelle valeur critique au del{à} de laquelle le syst{è}me ne poss{è}de plus qu'un seul
{é}quilibre, attractif {é}galement. Il s'agit ici de {\em bifurcation col-noeud}. A partir
d'une valeur critique de l'entr{é}e (c.{à}.d. du param{è}tre de bifurcation) apparaissent
deux nouveaux {é}quilibres, l'un d'eux {é}tant un noeud attractif, l'autre {é}tant un col.
A la valeur critique, l'{é}quilibre n'est pas hyperbolique.

\subsection{Bifurcation fourche}

Le mécanisme illustré à la figure \ref{fig:regwatt} est un \og régulateur de Watt \fg.  Ce dispositif peut servir à mesurer une
vitesse de rotation à partir d'un pointeur fixé sur l'axe vertical, ou, et
c'est pour cela qu'il a été inventé, à réguler cette vitesse si le pointeur
est relié à une vanne d'alimentation du moteur faisant tourner
le dispositif.
\begin{figure}[htbp] 
   \centering
   \includegraphics[height=4cm]{regwatt} \hspace{2cm}
   \includegraphics[height=4cm]{photo-regwatt} 
   \caption{Régulateur de Watt}
   \label{fig:regwatt}
\end{figure}
On peut vérifier que les équations décrivant le mouvement du système s'écrivent :
\eqnn
\dot x_1 &=& x_2\\
\dot x_2 &=& u^2\cos x_1 \sin x_1 -k \sin x_1 -Kx_2
\eeqnn
où $x_1 = \theta$ est la position angulaire des pendules symétriques et $u$ est la vitesse de rotation.

Ce dispositif a un équilibre en $(x_1, x_2, u)=(0,0, \bar u)$ et, si $\bar
u^2 > k,$ un autre équilibre en $(\bar x_1 = \mbox{ arc }\cos \frac{k}{\bar u^2}, 0,
\bar u)$ avec $\bar x_1 \in [0,\frac{\pi}{2}]$.  En fait $(-\bar x_1, 0, \bar
u)$ est aussi un équilibre qui correspondrait à la permutation des deux
pendules, ce qui est (physiquement) impossible mais conceptuellement
possible, d'après les équations ci-dessus.

La matrice Jacobienne du système autour de l'équilibre $(0,0, \bar u)$
s'écrit :
$$
A = \bma{cc}0 & 1\\ \bar u^2-k & -K
\ema
$$
Cet équilibre est attractif pour $\bar u^2 < k$ et répulsif pour $\bar u^2>k$.  Pour $\bar u^2
= k$, l'équilibre n'est pas hyperbolique.

Autour des deux autres équilibres, la matrice Jacobienne devient :
$$
A = \bma{cc} 0 & 1\\ \frac{k^2}{\bar u^2} -\bar u^2 & -K \ema \mbox{ avec } \bar u^2 > k
\Rightarrow \bar u^4 > k^2
$$
Ces équilibres sont donc attractifs.  Le diagramme de bifurcation peut alors
s'illustrer comme indiqué à la figure \ref{fig:biffourche}.  Il s'agit d'une bifurcation de
type fourche.
\begin{figure}[htbp] 
   \centering
   \includegraphics[height=5cm]{biffourche} 
   \caption{Diagramme d'équilibres - bifurcation fourche}
   \label{fig:biffourche}
\end{figure}
 
\subsection{Généralisations}

Nous avons d{é}crit dans cette section les bifurcations
relatives {à} des syst{è}mes d'ordre deux d{é}pendant d'un param{è}tre (la valeur de $\bar
u$). Ces bifurcations sont caract{é}ris{é}es par la travers{é}e de l'axe imaginaire du
plan complexe par une valeur propre r{é}elle de l'approximation lin{é}aire ou par une
paire de valeurs propres complexes conjugu{é}es (bifurcation de Hopf).  Lorsqu'un
syst{è}me d'ordre plus grand que deux d{é}pend d'un param{è}tre variable, il est rare que
plus d'une valeur propre r{é}elle (ou plus d'une paire de valeurs propres complexes
conjugu{é}es) franchisse l'axe imaginaire pour la m{ê}me valeur du param{è}tre de
bifurcation. Ce que nous venons de d{é}crire s'observe d{è}s lors aussi, dans des espaces
de phase plus compliqu{é}s {à} visualiser, pour des syst{è}mes d'ordre
 sup{é}rieur.  

\newpage
\section{Exercices}
\markboth{{\bf \hspace*{5mm} Chapitre 8}\hfill
Systèmes plans}
{{\bf Sec. \thesection}\hfill Exercices
\hspace*{5mm}}
 
\begin{exercice} {\bf \em Un système mécanique}

On considère un robot manipulateur à un segment relié à un
chassis  fixe par une articulation rotoïde. Le robot se déplace dans un
plan vertical. Il est actionné par un moteur produisant un couple
appliqué
à l'articulation et est soumis à un couple de frottement visqueux. La
flexibilité est négligée.
\begin{enumerate}
\item Etablir le modèle d'état du système.
\item Déterminer les configurations d'équilibre.
\item Analyser le comportement des trajectoires au voisinage des
équilibres en cas de frottement visqueux linéaire quand le couple
appliqué est constant.
\item Que peut-on dire des équilibres quand le frottement visqueux
est quadratique ?
\end{enumerate}
\end{exercice}
\vv

\begin{exercice}{\bf \em Un réacteur chimique}

Soit un réacteur continu parfaitement mélangé et à volume
constant dans lequel se déroule une réaction chimique 
irréversible mettant en oeuvre deux espèces $A$ et $B$~:
\eqnn
A \longrightarrow B.
\eeqnn
Le réacteur est alimenté uniquement avec l'espèce $A$, à
débit volumique constant strictement positif. La variable d'entrée est
la concentration d'alimentation du réacteur. La cinétique de réaction
est une fonction des concentrations des deux espèces : $r(x_A,x_B)$. 
\begin{enumerate}
\item Etablir le modèle d'état du système.
\item Montrer que, à entrée constante,  l'équilibre est unique et
stable si la cinétique obéit à la loi d'action des masses avec inhibition
hyperbolique par le produit.  Est-ce un noeud ou un foyer ?
\item Montrer que le système peut avoir des équilibres instables 
si la cinétique est une fonction monotone croissante de ses
arguments. 
\end{enumerate}
\end{exercice}
\vv

\begin{exercice} {\bf \em Un système à compartiments}

Quelles sont les conditions sur la structure du graphe d'un
système linéaire à deux compartiments pour que le système ait
une ou deux valeur propres nulles ? Quel est alors le comportement du
système (détailler les différents cas possibles) ?
\end{exercice}
\vv

\begin{exercice}{\bf \em Génératrice DC avec auto-excitation}

On considère une génératrice DC avec auto-excitation. La tension induite est, à vitesse constante, une fonction {\em monotone
croissante bornée} du courant d'excitation $E(I_s)$ telle que $E(0) >
0$. La génératrice débite sur une charge résistive. L'entrée de
commande du système est la vitesse de rotation de la
génératrice.
\begin{enumerate}
\item Déterminer le modèle d'état du système.
\item Montrer qu'on peut choisir le sens de référence des courants pour que le système soit positif.
\item Quelle allure doit avoir la fonction $E(I_s)$ pour qu'il y ait trois
équilibres hyperboliques isolés à vitesse de rotation constante. Discuter la
stabilité de ces équilibres.
\item Etudier les bifurcations de la configuration d'équilibre en fonction de la vitesse de rotation.
\end{enumerate}
\end{exercice}
\vv

\begin{exercice} {\bf \em Circuit électrique RLC} 

On considère le circuit électrique linéaire suivant :
\begin{figure}[h] 
   \centering
   \includegraphics[height=25mm]{E8-5} 
   \caption{Circuit électrique RLC}
   \label{fig:E8-5}
\end{figure}

où $R_2 = 1\Omega, C = 1F$ et $L = 1H$.

\begin{enumerate}
\item Ecrire un modèle d'état.
\item Déterminer les équilibres.
\item Quelles sont les conditions sur $R_1$ pour que chaque équilibre soit un
noeud, un foyer ou un col?
\end{enumerate}

On considère le même circuit électrique mais avec $R_1 = 1\Omega$ et $R_2$ une résistance non
linéaire décrite par la relation tension-courant $v_r = i^3_r - 3i^2_r + i_r$
\begin{enumerate}
\item Calculer les équilibres du système.
\item Caractériser le comportement du système au voisinage de ces équilibres.
\end{enumerate}
\end{exercice}
\vv 

\begin{exercice}{\bf \em Modélisation d'une activité de pêche.}

Dans un lac vit une espèce de poissons dont la croissance obéit à une loi logistique. Les poissons sont capturés par des pêcheurs suivant un principe d'action des masses. Les pêcheurs sont attirés vers le lac avec un taux directement proportionnel à la quantité de poissons dans le lac. Par contre les  pêcheurs sont découragés de pêcher avec un taux directement proportionnel au nombre de pêcheurs déjà présents.
\begin{enumerate}
\item Etablir un modèle d'état du système.
\item Etudier l'existence et la stabilité des états d'équilibre.
\end{enumerate}
\end{exercice}


\end{document}
