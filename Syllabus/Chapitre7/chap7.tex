\ifx \globalmark \undefined %% This is default.
	\documentclass[twoside,openright,11pt,a4paper]{report}

%\compiler avec xelatex
%\usepackage[applemac]{inputenc}
\usepackage[T1]{fontenc}
\usepackage[utf8]{inputenc} %latin1 est possible
%\usepackage[latin1]{inputenc} %latin1 est possible
\usepackage[francais]{babel}
\usepackage{lettrine}

\usepackage[text={13cm,20cm},centering]{geometry}

\renewcommand{\familydefault}{cmss}

\usepackage{graphicx}
\usepackage{amsmath}
\usepackage{amsfonts}
\usepackage{amssymb}
\usepackage{amsthm}
\usepackage{bm}
\usepackage{color}

\newcommand{\real}{\mathbb{R}}
\newcommand{\mb}{\mathbf}
\newcommand{\bos}{\boldsymbol}

\def \RR {I \! \! R}

\newcommand{\e}{\begin{equation}}  
\newcommand{\ee}{\end{equation}}
\newcommand{\eqn}{\begin{eqnarray}} 
\newcommand{\eeqn}{\end{eqnarray}} 
\newcommand{\eqnn}{\begin{eqnarray*}} 
\newcommand{\eeqnn}{\end{eqnarray*}} 

\newcommand{\bpm}{\begin{pmatrix}}
\newcommand{\epm}{\end{pmatrix}}

%\newcommand{\{\c c}}{\c c}

\newcommand{\bma}{\left(\begin{array}}
\newcommand{\ema}{\end{array}\right)} 
\newcommand{\hh}{\hspace{2mm}}
\newcommand{\hd}{\hspace{5mm}}
\newcommand{\hu}{\hspace{1cm}}
\newcommand{\vv}{\vspace{2mm}}
\newcommand{\vd}{\vspace{5mm}}
\newcommand{\vm}{\vspace{-2mm}}
\newcommand{\teq}{\triangleq}
%\newcommand{\qedb}{\,$\Box$}
\newcommand{\blanc}{$\left. \right.$}
\newcommand{\frts}[2]%
         {\frac{{\textstyle #1}}{{\textstyle #2}}}

\newcommand{\bindex}[3]%
{
\renewcommand{\arraystretch}{0.5}
\begin{array}[t]{c}
#1\\
{\scriptstyle #2}\\
{\scriptstyle #3}
\end{array}
\renewcommand{\arraystretch}{1}
}

\theoremstyle{definition}
\newtheorem{exemple}{{\bf Example}}[chapter]
\newtheorem{theoreme}[exemple]{{\bf Theorem}}
\newtheorem{propriete}[exemple]{{\bf Property}}
\newtheorem{definition}[exemple]{{\bf Definition}}
\newtheorem{remarque}[exemple]{{\bf Remark}}
\newtheorem{remarques}[exemple]{{\bf Remarks}}
\newtheorem{lemme}[exemple]{{\bf Lemma}}
\newtheorem{hypothese}[exemple]{{\bf Hypothesis}}
\newtheorem{exercice}{{\bf Exercise}}[chapter]

\newcommand{\xqedhere}[2]{%
 \rlap{\hbox to#1{\hfil\llap{\ensuremath{#2}}}}}

\newcommand{\xqed}[1]{%
 \leavevmode\unskip\penalty9999 \hbox{}\nobreak\hfill
 \quad\hbox{\ensuremath{#1}}}

\newcommand{\gf}{\fg\,\,}

\newcommand{\cata}[1] %
     {\renewcommand{\arraystretch}{0.5}
     \begin{array}[t]{c} \longrightarrow \\ {#1} \end{array}
     \renewcommand{\arraystretch}{1}}

\usepackage[isu]{caption}
%\usepackage[font=small,format=plain,labelfont=bf,up,textfont=it,up]{caption}
\setlength{\captionmargin}{60pt}

\newcommand{\cqfd}
{%
\mbox{}%
\nolinebreak%
\hfill%
\rule{2mm}{2mm}%
\medbreak%
\par%
}

\pagestyle{headings}

\renewcommand{\sectionmark}[1]{%
\markright{\thesection.\ #1}{}}

\renewcommand{\chaptermark}[1]{%
\markboth{\chaptername\ \thechapter.\ #1}{}}

\makeatletter 
\def\@seccntformat#1{\csname the#1\endcsname.\;} 
\makeatother

\title{ {\Huge {\textbf{Modélisation et analyse  \\ \vspace{4mm} des systèmes dynamiques }}} \\ \vspace{4cm} G. Bastin}

\date{\today}
	\begin{document} %% Crashes if put after (one of the many mysteries of LaTeX?).
\else 
	\documentclass{standalone}
	\begin{document}
\fi

\graphicspath{ {Chapitre7/images/} }

\setcounter{chapter}{6}
\chapter{Equilibria and invariants}
\chaptermark{Equilibria and invariants}\label{eqinv}



\lettrine[lines=1]{\bf I}{}n chapters 7, 8 and 9, we are going to study the dynamic systems behaviour $\dot x= f(x,u)$ when input variables are constant. In this particular chapter, we first take a look at the existence condition of equilibrium states and invariant subsets in the state space.

\section{Equilibria : definition and examples} 

\begin{definition}{\bf{\em Equilibrium}}

The pair $(\bar
x,\bar u)$ is an {\em \'equilibrium} of the system $\dot x = f(x,u)$ if
\eqnn
  f(\bar x,\bar u)=0. \xqedhere{5.5cm}{\qed}
\eeqnn
\end{definition}
This definition implies that if input signals are constant since $t_0$~:
\eqnn
u(t)=\bar u\;\;\;\forall t \geq t_0
\eeqnn
and if the system state
equals $\bar x$ at $t_0$~:
\eqnn
x(t_0)=\bar x
\eeqnn
then the system state stays constant and is equal to $\bar x$ for all later moments:
\eqnn
x(t)=\bar x\;\;\;\forall t \geq t_0.
\eeqnn
In some works, especially those related to  the process engineering,
an equilibrium is also called a {\em steady state}.
Similarly, the state $\bar x$ of equilibrium $(\bar x, \bar u)$ is sometimes called equilibrium point or {\em steady state} solution.


\begin{definition}{\bf{\em Isolated equilibrium}}

The pair $(\bar x,\bar u)$ is an {\em isolated} equilibrium 
if, for
$\bar u$ set, there exists a neighbourhood of $\bar x$ in $\real^n$ 
which contains only one vector $\tilde x$ such as $f(\tilde x,\bar
u)=0$. \qed
\end{definition}
Following examples show the large diversity of possible equilibrium configurations from
simple model systems characterized by balance equations.

\begin{exemple}{\bf{\em Free-flowing tank}}
 
 We consider a tank with constant section supplied by a pump whose volumetric flow
 $u$ is the input variable whereas the flow is free. (Fig.\ref{fig:diagecouli}). 
\begin{figure}[h] 
\begin{center}
\includegraphics[width=6cm]{reseecouli}
%\end{center} \end{figure}
%\begin{figure}[ht] 
%\begin{center}
\hspace{1cm}
\includegraphics[width=4cm]{diagecouli}
\caption{(a) Free-flowing tank (b) Equilibrium diagram}
\label{fig:diagecouli}
\end{center} 
\end{figure}

The system state model was first shown in chapter 4~:
\eqnn
\dot x = - \frac {kx\sqrt{x} }{S \beta + x} +u,
\eeqnn
where $x$ represents the volume of liquid in the tank.
The system equilibria satisfy the relation $k \bar x \sqrt{\bar
x} = \bar u (S \beta + \bar x)$ whose graph in $\real^2$ is called {\em
equilibrium diagram} (Fig. \ref{fig:diagecouli}).
We can see on this graph a distinct equilibrium state $\bar x$ for each distinct value $\bar u \geq 0$ and all equilibria are isolated. \qed
\end{exemple}
\vv

\begin{exemple}{\bf{\em Forced flow tank}}

Let's now suppose we still have the same tank as earlier, 
but this time the flow is forced by a pump 
whose flow rate $F_0$ is constant (Fig. \ref{fig:diagecoufor}).
\begin{figure}[ht]
\begin{center}
\includegraphics[width=6.5cm]{resecoufor}
\hspace{1cm}
\includegraphics[width=40mm]{diagecoufor}
\caption{(a) Forced flow tank. (b) Equilibria diagram.}
\label{fig:diagecoufor}
\end{center} 
\end{figure}
The state model becomes~:
$$\dot x=-F_0 + u.$$
As in previous example, the system reaches equilibrium when the input
flowrate matches exactly the output flowrate~:
$$\bar u=F_0.$$
This time, there is only one possible input value $u$ for which we have an equilibrium.
However, the equilibrium state
$\bar x$ can take any positive value. The equilibrium diagram is shown in figure \ref{fig:diagecoufor}. We observe that the equilibria are not isolated since $\bar x$ is 
undetermined. \qed
\end{exemple}
\vv

\begin{exemple}{\bf{\em Continuous stirred-tank reactor}}

Let's consider a continuous stirred-tank reactor with constant volume $V$ (Fig.\ref{fig:diagcuvmel}(a)). 
\begin{figure}[ht] 
\begin{center}
\includegraphics[width=60mm]{cuvmelvolcon}
\hspace{1cm}
\includegraphics[width=35mm]{diagcuvmel}
\caption{(a) Continuous stirred-tank reactor. (b) Equilibria diagram.}
\label{fig:diagcuvmel}
\end{center} 
\end{figure}
The feed rate carries a substance in solution (e.g., a dye) with concentration $x_{in}$. The feed rate $F$ is controlled by a valve~:
$$F=k u+b\;\;\;k>0,\;\;b>0,$$
where $u$ denotes the opening of the valve.

The system state is the dye concentration $x$ in the reactor and the state model is~:
$$\dot x= (x_{in}-x)\frac{k u+b}{V}.$$

The system is at equilibrium when the mass flow rate of the dye is exactly the same as the output mass flow rate~:
$$\frac{k \bar u +b}{V}x_{in}=\frac{k \bar u +b}{V}\bar x,$$
this implies $\bar x =x_{in}$.
The equilibrium diagram shown in figure \ref{fig:diagcuvmel}(b) shows that the equilibrium state is fixed and isolated but the corresponding constant input is undetermined. \qed
\end{exemple}
\vv

\begin{exemple}{\bf{\em Forced flow mixing tank}}

Up until now, we have only considered examples where the state vector has dimension 1. 
In systems of higher dimension, the different configurations described earlier can coexist
as we now illustrate with the example of a forced flow mixing tank (Fig. \ref{fig:cuvecoufor}).
\begin{figure}[ht]
\begin{center}
\includegraphics[width=7cm]{cuvecoufor}
\caption{Forced flow mixing tank}
\label{fig:cuvecoufor}
\end{center} 
\end{figure}
This system state model, noting $x_1$ the volume of the tank and $x_2$
the dye concentration in the tank, is~:
\begin{equation*} \begin{split}
\dot x_1&= -F_0+k u+b,\\
\dot x_2&=(x_{in}-x_2)\frac{k u+b}{x_1}.
\end{split} \end{equation*}
In this case, the equilibirum diagram has 3 dimensions (Fig.~\ref{fig:diagcuvemelcoufor}) 
and we can see there is only one input value giving an equilibrium, $\bar u=(F_0-b)/k$, 
and for this value $\bar u$, the volume of the equilibrium $\bar x_1$ is 
undetermined while the concentration at equilibrium is $\bar x_2=x_{in}$. \qed
\begin{figure}[h]
\begin{center}
\includegraphics[width=6cm]{diagcuvemelcoufor}
\caption{Equilibrium diagram for the forced flow mixing tank}
\label{fig:diagcuvemelcoufor}
\end{center} 
\end{figure}
\end{exemple}

Examples with 1 or 2 dimensions considered so far have shown situations where
\begin{itemize}
\item either the system has an isolated equilibrium for 
each input value $\bar u$,
\item or the system has an infinity of non-isolated equilibria
corresponding to a precise value of $\bar u$.
\end{itemize}
For non linear systems, other configurations are possible. In particular,
we can observe a few isolated equilibria corresponding to a same value of 
$\bar u$ as in the following example.

\begin{exemple}{\bf{\em Chemical reactor}}

Let's consider a continuous stirred-tank reactor where an irreversible exothermic reaction occurs 
$A \longrightarrow B$. The state model is (read Chapters 1 and 5)~:
\begin{equation*} \begin{split} 
\dot x_A &= -kx_A e^{-\frac{\alpha}{T}}+D(x_A^{in}-x_{A}),\\
\dot x_{B} &= kx_{A} e^{-\frac{\alpha}{T}}-D x_{B},\\
\dot T &= hkx_{A} e^{-\frac{\alpha}{T}}-qT+u,
\end{split} \end{equation*}
where $x_{A}$ and $x_{A}^{in}$ are the concentrations of reactant $A$
in the reactor and in the supply, $x_B$ is the concentration of product $B$, $D$ is the constant volumetric flow rate,
$T$ is the temperature and $u$ the heat input per unit of time.

The equilibria of this system are characterized by those equations
\begin{equation*} \begin{split}
\bar x_{A}&=\frac{Dx_{A}^{in}}{k e^{-\alpha/\bar T}+D}, \\
\bar x_{B}&=\frac{k \bar x_{A} e^{-\alpha/\bar T}}{D}, \\
\bar T&=\frac{1}{q}\left(\frac{Dx_{A}^{in}hk  e^{-\alpha/\bar T}}{k
 e^{-\alpha/\bar T}+D} + \bar u \right).
\end{split} \end{equation*}
The third equation allows us to determine $\bar T$ as a function of $\bar u$. The first two then allow us to deduce from $\bar T$ values of equilibrium for $\bar x_{A}$ and $\bar x_{B}$. 
\begin{figure}[h]
\begin{center}
\includegraphics[height=4.4cm]{eqS}
\caption{Equilibrium diagram for a simple chemical reactor}
\label{fig:eqS}
\end{center} 
\vspace{-5mm}
\end{figure}
The equilibrium diagram representing $\bar T$ as a function of $\bar u$ is illustrated in figure \ref{fig:eqS}.
According to the values of $\bar u$, we see that there is one, two or three isolated equilibria.\qed
\end{exemple}

\section{Equilibria of linear systems} 

Let the linear system
\eqnn
\dot x = Ax + Bu \label{syslin}
\eeqnn
for which the equation defining the equilibria becomes
\eqnn A \bar x + B \bar u = 0. \eeqnn
The equilibria of a linear system are totally 
characterized by the following therorem.

\begin{theoreme}{\blanc}
\begin{itemize}
\item  If matrix $A$ is regular, then for each $\bar u$, the pair
$(-A^{-1}B\bar u,\bar u)$ is an isolated equilibrium.
\item If matrix $A$ is singular, the system
(\ref{syslin}) has an infinity of equilibria (non-isolated) 
provided that $B\bar u \in \mbox{Im}A$. Those equilibria are an 
affine manifold solution of system $A \bar x = -B \bar u$. On
the other side, for each $\bar u$ such as $B\bar u \notin \mbox{Im}A$, the 
system (\ref{syslin}) does not have an equilibrium. \qed
\end{itemize}
\end{theoreme}

For dynamic linear systems, we can't have several isolated equilibria corresponding 
to the same input value $\bar u$. 
Finally, note that the pair $(\bar x, \bar u) = (0,0)$ is always an equilibrium 
for dynamic linear systems of the form (\ref{syslin}).

\begin{exemple}{\bf{\em Linear models of DC machines}}

Several models of direct current machines (motors and generators) 
were introduced in section~3.6. Under general assumptions of linear 
viscous friction and nonstauration of the streams, some of those models 
are linear. We examine below their configuration of equilibrium.\\

\noindent{\em DC generator driven by the stator}

On consid\`ere le mod\`ele d'\'etat d'une g\'en\'eratrice \`a courant continu tournant \`a {\it vitesse $\omega$ constante}. En notant $x_1=I_s$, le courant statorique, $x_2=I_r$, le courant rotorique et $u=v_s$ la tension aux bornes du circuit statorique, le mod\`ele d'\'etat est lin\'eaire et s'\'ecrit comme suit~:
$$
\bma{c}\dot x_1\\ \dot x_2
\ema=\bma{ccc}-\dfrac{R_s}{L_s} & & 0\\ & \vspace{-2mm} & \\ K_e\omega & &-\dfrac{R_r+R_L}{L_r}\ema
\bma{c}x_1\\x_2 \ema + \bma{c} \dfrac{1}{L_{s}}\\ \vspace{-2mm} \\ 0\ema u
$$
The matrix $A$ of this linear system is invertible and the 
generator therefore has an isolated equilibrium state for 
each value of the input voltage $\bar u$~:
\eqnn
\bar x_1&=&\dfrac{L_s}{R_s}\bar u\\
\bar x_2&=&\dfrac{L_r}{R_r+R_L}\dfrac{L_s}{R_s}K_e\omega \bar u
\eeqnn

\noindent{\em DC motor controlled by the rotor}

With state variables for this system
$x_1=\theta$, angular position of the rotor, $x_2=\dot \theta=\omega$, angular velocity and $x_3=I_r$, the rotor current, the following model is obtained~:
$$\bma{c}\dot x_1\\ \dot x_2\\ \dot x_3\ema =
\bma{ccc}0&1&0\\ $ \vspace{-2mm} $ \\ 0&-\dfrac{B}{J}&\dfrac{K_mI_s}{J}\\ $ \vspace{-2mm} $ \\
0&-\dfrac{K_e I_s}{L_r} &-\dfrac{R_r}{L_r}\ema \bma{c}x_1\\x_2\\x_3\ema
+\bma{cc}0&0\\ &\vspace{-2mm} \\ 0&\dfrac{1}{J}\\ &\vspace{-2mm} \\ \dfrac{1}{L_r}& 0\ema \bma{c}u_1\\u_2\ema
$$
where $u_1$ is the resisting torque and 
$u_2$ the control voltage.
We observe that :
\begin{itemize}
\item the state matrix $A$ of the system is singular,
\item $B\bar u =(0\;\; \bar u_2/J\;\;\bar u_1/L_r)^T \notin \mbox{Im} A$
unless $\bar u_1/\bar u_2=-R_r/K_mI_s$ or if $\bar u_1=\bar u_2=0$.
\end{itemize}
The first case corresponds to a rotor control voltage which creates 
a torque exactly compensating for the load torque. The rotational speed 
is then zero and the rotor angular position is undetermined. The 
equilibrium value of the rotor current is given by $\bar x_3=\bar I_r=\bar u_2/K_m I_s$.
In the second case, the equilibria are of the form $\bar x_1=\bar \theta, \bar
x_2=0, \bar x_3=0$, i.e. that the engine is stopped with the rotor in any angular position.

We can also consider equilibria of the subsystem
whose states are the speed $\omega$ and the current $I_r$~:
$$\bma{c} \dot x_2\\ \dot x_3\ema =
\bma{cc}-\dfrac{B}{J}&\dfrac{K_m I_s}{J}\\ & \vspace{-2mm} \\
-\dfrac{K_e I_s}{L_r} &-\dfrac{R_r}{L_r}\ema \bma{c}x_2\\x_3\ema
+\bma{cc}0&\dfrac{1}{J}\\ & \vspace{-2mm} \\ \dfrac{1}{L_r}& 0\ema \bma{c}u_1\\u_2\ema.
$$ 
The state matrix of this system is invertible (all constants are positive
and the determinant is not zero) and each value of the input vector $\bar u$
will correspond to an equilibrium value of the state vector $(\bar
x_1\;\;\bar x_2)^T$. This equilibrium, which is not
contradictory with the previous one, corresponds to the case of a DC motor which
drives a load by rotating at constant speed. \qed
\end{exemple}

\section{Invariants}

The notion of invariant we will define in this section is a generalization of the concept of equilibrium.

\begin{definition}{\bf{\em Invariant}}

The subset $\mathcal{X} \times U \subset \real^n \times \real^m$ is an invariant
of the dynamic system $\dot x = f(x,u)$ si :
\eqnn
\left\{
\begin{array}{l}
x(t_0) \in \mathcal{X}\\
u(t) \in U \;\;\;\forall t \geq t_0 
\end{array}\right\} & \Rightarrow & \left\{\begin{array}{l} 
x(t) \text{ existe}\\ x(t) \in \mathcal{X} \end{array} \right\}\forall t \geq t_0 \xqedhere{2cm}{\qed} 
\eeqnn
\end{definition}

Thies definition means that if the state of the system is in $\mathcal{X}$ at
an initial time, it will remain for all subsequent moments as long as the input
signal $u(t)$ will be maintained in $U$.

We have already seen several examples of invariants in the previous chapters. The simplest example is the set of equilibria of a system corresponding to a constant input $\bar u$.  

In this case, the subset $U = \{\bar u\}$ is reduced to a singleton and $\mathcal{X}$
contains the corresponding equilibrium states $\bar x$.  

Another typical exampe is the{\em positive orthant} $(\mathcal{X} = \real^n_+) \times (U \subset \real^m)$
which is, by definition, an invariant for positive systems 
(see Definition 4.3 and Theorem 4.4.).

There are various ways to characterize the invariants of a dynamical system depending on the particular form if the subset $\mathcal{X}$.  We will show two remarkable characterizations : in the first one $\mathcal{X}$
is an open subspace of $\real^n$, in the second one $\mathcal{X}$ is an
hypersurface in $\real^n$.  \\

\noindent{\bf $\bullet$ \!$\mathcal{X}$  is an open subspace in $\real^n$}

Let $\mathcal{X}$ an open subset in $\real^n$ whose border $\partial \mathcal{X}$ is regular enough.  If at any point $y$ on the border $\partial \mathcal{X}$, the vector $f(y,v)$ points inwards $\mathcal{X}$ for all $v \in U$, then the subset $\mathcal{X} \times U$
is an invariant of the system $\dot x = f(x,u)$.

This characterization will be discussed further in chapter 8 (section 8.4).\\

\noindent{\bf $\bullet$ \!$\mathcal{X}$ is an hypersurface in $\real^n$}

We call {\it first integral} a function $z = h(x)$ of class $C^2$ such that~:
\eqn
\frac {\partial h}{\partial x}f(x,u) = 0 \;\;\; \forall u \in U. \label{condit}
\eeqn
We define the subset $\mathcal{X}$ as follows :
\eqnn
\mathcal{X} \triangleq \{x \in R^n : h(x) = c \}
\eeqnn
with $c$ any real constant.
This set $\mathcal{X}$ is an hypersurface in $\real^n$.  As the condition (\ref{condit}) implies that the function $z = h(x)$ is constant along the trajectories, it is clear that the subset
$\mathcal{X} \times U$ is an invariant of the system $\dot x = f(x,u)$.  The
{\em reaction invariants} are a typical example. 

\begin{exemple}{\bf{\em Reaction invariants}}

As we saw in chapter 5, the model of a continuous stirred-tank reactor is written as follows~:
\eqn
\dot x = Cr(x) + u(x^{in} -x) \label{reacteur}
\eeqn
where $x$ is the composition of the reaction medium, $u$ the feed flow rate,
$x^{in}$ the composition (supposed constant) of the feed flow, $C$ is 
the stoichiometric matrix and $r(x)$ the vector of the reaction kinetics.

The flow $u$ is positive and bounded by the maximum capacity of the feed pump $u_{\max}$, so we define $U$ as the closed interval :
\eqnn
U = [0, u_{\max}]
\eeqnn
On the other hand, the subset $\mathcal{X}$ is defined as follows :
\eqnn
\mathcal{X} =\{x:x\in \real^n_+, \;\; Lx = Lx^{in}\}
\eeqnn
where $L$ is a matrix $(n-p)\times n$ such as $LC=0$.  In other words, the rows of the matrix are a base of the kernel of the stoichiometric matrix transposed $C$.

The subset $\mathcal{X} \times U$ thus defined is an invariant of the system
(\ref{reacteur}).  To check our assumptions, we consider the partial linear transformation of state :
\eqnn
z = Lx
\eeqnn
and we compute its evolution along the system trajectories :
\eqnn
\dot z = LCr(x) + u(Lx^{in} -Lx) =   u(Lx^{in} -Lx) \;\; \mbox{car } LC=0
\eeqnn
According to the definition of $\mathcal{X}$, we immediately observe that, if $Lx(t_0) =
Lx^{in}$, then $\dot z =0$ along the system trajectories and therefore
$Lx(t)= Lx^{in} \;\;\;\forall t \geq t_0$, independently of the input signal $u(t)$.

Moreover, the system (\ref{reacteur}) is a positive system, therefore
$x(t) \in \real^n_+$ $\forall t \geq t_0$ if $x(t_0) \in \real^n_+$ and if $u(t) \in U
\;\forall t \geq t_0$.

Invariants defined this way are called reaction invariants or chemical invariants in the literature. \qed
\end{exemple}


\section{Exercises}

\begin{exercice}{\bf \em An electromagnetic relay}

Determine the equilibria of the state model of an electromagnetic relayd in chapter 3, example 3.2 (see also exercise 6.2). \qed
\end{exercice}
\vv

\begin{exercice}{\bf \em Direct current generator}

We consider the model of a DC generator (see chapter 3, section 3.6) feeding into a resistive load with a linear viscous friction.
\begin{enumerate}
\item Find the equilibria based on inputs $\bar u_1$ and $\bar u_2$
\item Determine the optimal operating points that maximize the current supplied by the generator.. \qed
\end{enumerate}
\end{exercice}
\vv

% --------------------------------------------------------------------------------------------------------------
% Pierre-Alexandre's part starts here
\begin{exercice}{\bf \em A phase-locked loop}

A phase-locked loop (Phase-Locked Loop) used in communication networks is described by the equation
\eqnn
\ddot y + (a + b \cos y) \dot y + u \sin y = 0
\eeqnn
avec $a > b > 0$ et $u(t) > 0 \, \forall t$. 
\begin{enumerate}
\item Put the system as a state model.
\item Determine the equilibria.
\end{enumerate}
\end{exercice}
\vv

\begin{exercice}{\bf \em A boat}

Determine the equilibria of the state model of the boat for the exercise 2.7. What is the physical meaning of these equilibria ? \qed
\end{exercice}
\vv 

\begin{exercice}{\bf \em An industrial shredder}

\begin{figure}[htbp] 
   \centering
   \includegraphics[height=5cm]{broyeur} \hfill
	\includegraphics[height=5cm]{broyeur-photo}
   \caption{Grinding circuit - Photo of an industrial shredder}
   \label{fig:broyeur}
\end{figure}
The operation of an industrial grinding circuit (fig. \ref{fig:broyeur}) is expressed by the state model~:
\begin{equation*} \begin{split} 
\dot x_1 &= -\gamma_1 x_1 +(1-\alpha) \phi(x_3),\\
\dot x_2 &= -\gamma_2 x_2 +\alpha \phi(x_3),\\
\dot x_3 &= \gamma_2 x_2 - \phi(x_3)+u.
\end{split} \end{equation*}
with the following notations~:\\

\begin{tabular}{ll}
$x_1$&= quantity of end product in the separator;\\
$x_2$&= amount of recycled material in the separator;\\
$x_3$&= amount of material in the mill:\\
$u $&= mill feed rate.
\end{tabular}\\

\noindent The parameter$\alpha$ is the characteristic constant of separator. 
($0 < \alpha < 1$). The function of grinding  $\phi(x_3)$ has the following form~:
\eqnn
\phi(x_3) = k_1x_3e^{-k_2x_3}
\eeqnn
where $k_1$ et $k_2$ are positive constants.
\begin{enumerate}
\item Show that this is a compartment system and give the graphs of system.
\item Determine the equilibria of system.
\item The set that is described by the following inequalities represents a jam situation. Show it is an invariant system.
\begin{equation*} \begin{split} 
&  x_1 \geq 0, \hspace{5mm} x_2 \geq 0, \hspace{5mm} x_3 \geq 0, \\
&  (1 - \alpha)\phi(x_3) \leq \gamma_1x_1 < \bar u, \\ 
&  \alpha \phi(x_3) \leq \gamma_2x_2, \\ 
&  \frac{\partial\phi(x_3)}{\partial x_3} < 0. \xqedhere{6.5cm}{\qed}
\end{split} \end{equation*}
\end{enumerate}
\end{exercice}
\vv

\begin{exercice}{\bf \em A biochemistry reactor}

We consider the state model of a biochemistry reactor from exercise 6.2.
\begin{enumerate}
\item Determine the system equilibria and sketch equilibria diagrams.
\item Determine the invariants of the reactor.
\item Same questions if the reactor is reversible. \qed
\end{enumerate}
\end{exercice}
\vv

\begin{exercice}{Dynamics of a viral infection}

The dynamics of a viral infection with lytic and non-lytic actions of immunization  is described by the following model state:
\begin{equation*} \begin{split}
\dot x_1 &= \lambda - dx_1 - \dfrac{\beta x_1 x_2}{1 + q x_3}, \\
\dot x_2 &= \dfrac{\beta x_1 x_2}{1 + q x_3} - a x_2 - p x_2 x_3, \\
\dot x_3 &= c x_2 - b x_3.
\end{split} \end{equation*}
In these equations, $x_1$, $x_2$ et $x_3$ are respectively the quantities of healthy cells, infected and immune. The infected cells produce viral particles. $\lambda$ is the healthy cell production rate and $d$ their mortality rate. Lytic components of the anti-viral activity kill infected cells while non-lytic components inhibit replication of the viral particles. The infected cells are killed at the speed$p x_3$ with $p$ represents the intensity of anti-viral lytic activity. The production of infected cells is represented by the term
$$
\dfrac{\beta x_1 x_2}{1 + q x_3}
$$
where $q x_3$ represents the intensity of inhibition replication by the anti-viral non-lytic activity. The death rate of infected cells is $a$ and the death rate of immune cells is $b$. $c x_2$ is the production rate of immune cells.
\begin{enumerate}
\item Show that the model state is a reaction system.
\item Show that the model state is equivalent to a compartment systems.
\item Find the equilibria of the system in the positive orthant.
\end{enumerate}
\end{exercice}
\vv


\begin{exercice}{\bf \em Mechanical system}

We consider the modeling of a mechanical system with one degree of freedom~:
\eqnn
\ddot \theta + c \dot \theta + r \sin \theta = 0
\eeqnn
\begin{enumerate}
\item Wite the model state of the system ($x_1 = \theta$).
\item Determine the equilibria.
\item Show with the condition $c^2 \geq 4 r$, there exists an invariant bound (which interior is non-empty) in the orthant $\{ x_1 \geq 0, x_2 \leq 0 \}$. \qed
\end{enumerate}
\end{exercice}

\end{document}

